%%% LaTeX Template: Article/Thesis/etc. with colored headings and special fonts
%%%
%%% Source: http://www.howtotex.com/
% vim: set spell spelllang=es syntax=tex :

\documentclass[12pt]{article}
\usepackage{styles/apuntes-estilo}
\usepackage{fancyhdr,lastpage}
\usepackage{hyperref}
\usepackage[inline]{enumitem}

\def\maketitle{

\makeatletter
{\color{bl} \centering \huge \sc \textbf{ Trabajo práctico N° 2\\ \large
\vspace*{-8pt} \color{black} Representación de datos\vspace*{8pt} }\par}
\makeatother

\makeatletter
% vim: set spell spelllang=es syntax=tex :
 {\centering \small 
    Introducción a la computación\\
    Departamento de Ingeniería de Computadoras \\
    Facultad de Informática - Universidad Nacional del Comahue \\
    \vspace{20pt} }
\makeatother

\vspace{-2.5cm}
\mbox{\hspace{-1cm}\includegraphics[width=3cm,height=3cm]{logos/uncoma.pdf}\hspace{13cm}
    \includegraphics[width=2.9cm,height=2.9cm]{logos/fai.pdf}}



}

% Custom headers and footers
\fancyhf{} % clear all header and footer fields
\fancypagestyle{plain}{\fancyhf{}}
\pagestyle{fancy}
\lhead{\footnotesize TP N° 2 - Representación de datos}
\rhead{\footnotesize \thepage\ }

\def\ti#1#2{\texttt{#1} & #2 \\ }

\begin{document}

\thispagestyle{empty}
\maketitle
\setlength{\parindent}{1pt}

\textbf{Objetivo:} comprender la representación binaria de números enteros y
de punto (coma) flotante, y la suma de números enteros en binario. Comprender
la representación binaria de texto y otros datos más complejos.

\textbf{Bibliografía básica:}

\vspace{-2\topsep}
\begin{itemize}

    \itemsep2pt \parskip0pt \parsep0pt

    \item Apuntes de cátedra. Disponible en \it{PEDCO}.

    \item Andrew S. Tanenbaum. \emph{Organización de computadoras: un enfoque
        estructurado}. Tercera edición, México, Editorial Prentice Hall, 1992.
        ISBN 968-880-238-7. Disponible en biblioteca.

    \item Wikipedia: \emph{Coma flotante}.
        \url{http://es.wikipedia.org/wiki/IEEE_coma_flotante}
    
    \item Wikipedia: \emph{Conversión analógica digital}.
        \url{http://en.wikipedia.org/wiki/Analog-to-digital_converter}

\end{itemize}

\textbf{Recursos:}

\vspace{-2\topsep}
\begin{itemize}

    \itemsep2pt \parskip0pt \parsep0pt

    \item Calculadora IEEE-754: \url{http://www.zator.com/Cpp/E2_2_4a1.htm}

    \item Tabla de caracteres \emph{UTF-8}:
        \url{http://www.fileformat.info/info/charset/UTF-8/list.htm}

    \item Tabla de caracteres ASCII Extendida:
        \url{http://www.programasprogramacion.com/caracteres.php}

\end{itemize}

\textbf{Nota}: La abreviatura ``Hex'' significa Hexadecimal, y el prefijo
``\textbf{0x}'' indica que un número está en hexadecimal.


\section{Representación de números enteros}

\begin{enumerate}

    \item Completar la siguiente tabla en el sistema binario. Recuerde que,
        para la representación en 8 bits, debe completar con ceros a la
        izquierda en caso de ser necesario.

    \begin{center}

        \begin{tabular}[t]{|c|c|c|}

        \hline

            &\multicolumn{2}{|c|}{\textbf{Sistema Binario}}\\

        \hline

            \textbf{Sistema} & \textbf{Sin Signo}& \textbf{Sin Signo}\\

            \textbf{Decimal} & ~ & \textbf{en 8 bits}\\

        \hline

            0 & \hspace{14em}~&\hspace{14em}~\\

        \hline

            9&&\\

        \hline

            37&&\\

        \hline

            40&&\\

        \hline

            80&&\\

        \hline

            147&&\\

        \hline

            255&&\\

        \hline

        \end{tabular}

    \end{center}

\item ¿Cuál es la \emph{fórmula general} para obtener el rango de números
    representables para \textbf{n} bits si la representación se trata de:

    \begin{enumerate*}[itemjoin=\hspace{2em}]

        \item Sin signo?

        \item Signo magnitud?

        \item Complemento a 2?

    \end{enumerate*}

\item Para un sistema binario \emph{Sin Signo} ¿cuál es el rango de números
    representables con \textbf{8 bits}?

\item Completar la siguiente tabla con la representación en \emph{8 bits} de
    los siguientes números en \emph{Signo Magnitud} y \emph{Complemento a 2}.

    \begin{center}

        \begin{tabular}[t]{|c|c|c|}

        \hline

            \textbf{Decimal} & \textbf{Signo Magnitud} & \textbf{Complemento a
            2}\\

        \hline

            3 & \hspace{14em}~&\hspace{14em}~\\

        \hline

            -3&&\\

        \hline

            66&&\\

        \hline

            -66&&\\

        \hline

            90&&\\

        \hline

            -90&&\\

        \hline

            127&&\\

        \hline

            -127&&\\

        \hline

        \end{tabular}

    \end{center}

    \item ¿Cuál es el rango de números representables para \textbf{8 bits} en:

        \begin{enumerate*}[itemjoin=\hspace{2em}]

            \item Sin signo?

            \item Signo magnitud?

            \item Complemento a 2?

        \end{enumerate*}

    \item Complete la siguiente tabla que representa enteros de \textbf{4
        bits}.

        \begin{center}

            \begin{tabular}[t]{|c|c|c|c|}

            \hline

                &\multicolumn{3}{|c|}{\textbf{Sistema Binario}}\\

            \hline

                \textbf{Sistema} & \textbf{Complemento a 2}& \textbf{Signo
                Magnitud} & \textbf{Sin Signo}\\

                \textbf{Decimal} & ~ & ~ &\\

            \hline

                7 & \hspace{9em}~&\hspace{9em}~&\hspace{9em}~\\

            \hline

                6&&&\\

            \hline

                5&&&\\

            \hline

                4&&&\\

            \hline

                3&&&\\

            \hline

                2&&&\\

            \hline

                1&&&\\

            \hline

                0&&&\\

            \hline

                -1&&&\\

            \hline

                -2&&&\\

            \hline

                -3&&&\\

            \hline

                -4&&&\\

            \hline

                -5&&&\\

            \hline

                -6&&&\\

            \hline

                -7&&&\\

            \hline

                -8&&&\\

            \hline

            \end{tabular}

        \end{center}

        \begin{enumerate}
                
            \item Una vez completada la tabla, a cada valor de la columna
                Complemento a 2 aplique la operación de complemento a 2 y
                responda: ¿Cuál es el \emph{significado aritmético} de lo que
                observamos?

            \item ¿Cuál es el rango de números representables para \textbf{4
                bits} en:

            \begin{enumerate*}[itemjoin=\hspace{2em}]

                \item Sin signo?

                \item Signo magnitud?

                \item Complemento a 2?

            \end{enumerate*}

    \end{enumerate}

    \item Representar en Complemento a 2 los siguientes números enteros
        decimales. Utilizar representaciones de 8, 16 o 32 bits según sea
        necesario.

        \begin{center}

            \begin{tabular}[t]{|c|c|}

            \hline

                \textbf{Sistema} & \textbf{Complemento a 2}\\

                \textbf{Decimal} & ~ \\

            \hline

                -50 & \hspace{27em}~ \\

            \hline

                -128&\\

            \hline

                -256&\\

            \hline

                -542&\\

            \hline

                -40090&\\

            \hline

            \end{tabular}

        \end{center}

    \item Complete la siguiente tabla para los números hexadecimales
        representados en 8 bits. Una vez expresado en numero hexadecimal en
        binario, interprete la secuencia de bits como un numero decimal
        expresado en los sistemas \emph{Sin signo} y \emph{Complemento a 2}.

        \begin{center}

            \begin{tabular}[t]{|c|c|c|c|}

            \hline

                \textbf{Hex.} & \textbf{Binario} & \textbf{Sin Signo} & \textbf{Complemento a 2}\\

            \hline

                \textbf{2B} & \hspace{9em}~ &~&~\\

            \hline

            \textbf{9F}&&&\\

            \hline

            \textbf{F9}&&&\\

            \hline

            \textbf{5D}&&&\\

            \hline

            \textbf{E4}&&&\\

            \hline

            \end{tabular}

        \end{center}

    \item En \emph{Complemento a 2} ¿cuál es el entero negativo que puede ser
        representado y cuyo opuesto positivo no, si se utiliza:

        \begin{enumerate*}[itemjoin=\hspace{2em}]

            \item 4 bits?

            \item 8 bits?

            \item 16 bits?

        \end{enumerate*}

    \item Indicar el rango de los números representables con 4, 8, 16 y 32
        bits utilizando notación:

        \begin{center}

            \begin{tabular}[t]{|c|c|c|c|}

            \hline

                 & \textbf{Sin Signo} & \textbf{Complemento a 2} &
                 \textbf{Signo Magnitud}\\

            \hline

                4 bits & \hspace{9em}~ &\hspace{9em}~&\hspace{9em}~\\

            \hline

            8 bits&&&\\

            \hline

            16 bits&&&\\

            \hline

            32 bits&&&\\

            \hline

            \end{tabular}

        \end{center}

\end{enumerate}

\section{Operaciones aritméticas de números enteros}

\begin{enumerate}

    \item Dados los siguientes números representados en Complemento a 2 con 6
        bits, efectuar las siguientes restas:
    
        \begin{enumerate*}[itemjoin=\hspace{2em}]

            \item $00\,1010 - 00\,0110$

            \item $01\,0000 - 00\,0001$

            \item $01\,1100 - 11\,1111$

        \end{enumerate*}

    \item Determinar cuáles de las siguientes operaciones producen overflow,
        considerando una representación en \emph{complemento a 2} con
        \textbf{8 bits}:

        \begin{enumerate*}[itemjoin=\hspace{2em}]

            \item $0100\,1111 + 0011\,1100$

            \item $0101\,1111 + 1011\,1100$

            \item $1010\,0100 + 1101\,1000$

        \end{enumerate*}

    \item Elija un numero $N$ entre 33 y 50 y complete la siguiente tabla,
        realizando la división entera:

        \begin{center}

            \begin{tabular}[t]{|c|c|c|}

            \hline

                 & \textbf{Decimal} & \textbf{Binario} \\

            \hline

                $N$ & \hspace{9em}~ &\hspace{9em}~\\

            \hline

                $N/(2^1)$ & ~ &~\\

            \hline

                $N/(2^2)$ & ~ &~\\

            \hline

                $N/(2^3)$ & ~ &~\\

            \hline

                $N/(2^4)$ & ~ &~\\

            \hline

                $N/(2^5)$ & ~ &~\\

            \hline

            \end{tabular}

        \end{center}

        \begin{enumerate}

            \item ¿Puede deducir algún mecanismo sencillo para dividir por dos un
                numero representado en binario?

            \item ¿Puede deducir algún mecanismo sencillo para multiplicar por dos un
                numero representado en binario?

            \item ¿Ocurre algo similar en otras bases? (considere como
                multiplicar y dividir por 10 en base 10, sin realizar cuentas)

        \end{enumerate}

\end{enumerate}

\section{Representación de números reales}

\begin{enumerate}

    \item Representar los números reales en notación de \emph{punto Fijo} y
        \emph{complemento a 2}, utilizando 4 bits para la parte entera y 4
        para la parte fraccionaria:

        \begin{enumerate*}[itemjoin=\hspace{2em}]

            \item $1.75$

            \item $-1.75$

            \item $7.0625$

            \item $-5.9$

            \item $-4.5$

            \item $3.9$

        \end{enumerate*}

    \item Los siguientes números están representados en \emph{Punto Flotante
        IEEE-754 de precisión simple (32 bits)}. Indique a qué número decimal
        se corresponde:

        \begin{enumerate*}[itemjoin=\hspace{2em}]

            \item 0x41700000

            \item 0x42CD8000

            \item 0x42C68000

            \item 0x42008000

        \end{enumerate*}

    \item Los siguientes números están representados en \emph{Punto Flotante
        IEEE-754 de precisión doble (64 bits)}. Indique a qué número decimal
        se corresponde:

        \begin{enumerate*}[itemjoin=\hspace{2em}]

            \item 0x4055F9999999999A

            \item 0x4059C7AE147AE148

        \end{enumerate*}

    \item Convertir del sistema decimal a la notación \emph{Punto Flotante
        IEEE-754 de precisión simple (32 bits)} y mostrar el resultado final
        en notación hexadecimal:

        \begin{enumerate*}[itemjoin=\hspace{2em}]

            \item $65.375$

            \item $-0.5$

            \item $-5.6$

            \item $100.003$

            \item $19.14$

        \end{enumerate*}

    \item Para cada inciso del ejercicio anterior, realice la conversión
        inversa (es decir, de Punto Flotante a expresión decimal) e indique el
        \textbf{error de precisión cometido}.

    \item ¿Cómo se representa el 0 en notación \emph{Punto Flotante IEEE-754}?
        Calcular y luego investigar.

    \item Calcular el rango de los números reales representables con el
        formato \emph{IEEE-754 de precisión simple}.

\end{enumerate}

\section{Codificación de texto}

\begin{enumerate}

    \item Decodifique los siguientes mensajes codificados en \emph{UTF-8} y
        representados en hexadecimal.

        \begin{enumerate}

            \item \textbf{41 79 75 64 61}

            \item \textbf{45 6C 20 C3 B1 61 6E 64 C3 BA 20 62 61 6A C3 B3 20
                65 6C 20 C3 A1 72 62 6F 6C}

            \item \textbf{48 6F 6C 61 20 6D 75 6E 64 6F}

            \item Para cada uno de los mensajes anteriores, responda: ¿cuántos
                caracteres posee? ¿cuántos bytes ocupa?

        \end{enumerate}

\end{enumerate}

\section{Elaboración de texto}

\begin{enumerate}

    \item Sabiendo que la tarjeta \textit{SUBE} debe almacenar el crédito
        disponible con un rango de $[-20:600]$ pesos ¿Que sistema de
        representación propendía usted? ¿Con cuantos bytes recomendaría?
        ¿Puede el sistema propuesto representar un saldo de $\$0.01$ pesos?
        Justifique sus elecciones.

    \item Analice la siguiente afirmación: ``\emph{Si se complementa a 2 un
        número entero representado en completo a 2, el resultado siempre es su
        opuesto}'' ¿Verdadero o falso? Justifique su respuesta.

\end{enumerate}

\end{document}
