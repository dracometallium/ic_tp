%%% LaTeX Template: Article/Thesis/etc. with colored headings and special fonts
%%%
%%% Source: http://www.howtotex.com/
% vim: set spell spelllang=es syntax=tex :

\documentclass[12pt]{article}
\usepackage{styles/apuntes-estilo}
\usepackage{styles/egyptian}
\usepackage{fancyhdr,lastpage}
\usepackage{hyperref}
\usepackage[inline]{enumitem}
\usepackage{xurl}

\def\maketitle{

\makeatletter{
    \color{blue} \centering \huge \sc
    \textbf{
        Trabajo práctico N° 2\\
        \large \vspace*{-8pt} \color{black}
        Unidades de información
        \vspace*{8pt}
    }\\
    \small Fecha de finalización: 5 de abril
    \par
}

\makeatother

\makeatletter
% vim: set spell spelllang=es syntax=tex :
 {\centering \small 
    Introducción a la computación\\
    Departamento de Ingeniería de Computadoras \\
    Facultad de Informática - Universidad Nacional del Comahue \\
    \vspace{20pt} }
\makeatother

\vspace{-2.5cm}
\mbox{\hspace{-1cm}\includegraphics[width=3cm,height=3cm]{logos/uncoma.pdf}\hspace{13cm}
    \includegraphics[width=2.9cm,height=2.9cm]{logos/fai.pdf}}



}

% Custom headers and footers
\fancyhf{} % clear all header and footer fields
\fancypagestyle{plain}{\fancyhf{}}
\pagestyle{fancy}
\lhead{\footnotesize TP N° 2 - Unidades de información}
\rhead{\footnotesize \thepage\ }

\def\ti#1#2{\texttt{#1} & #2 \\ }

\begin{document}

\thispagestyle{empty}
\maketitle
\setlength{\parindent}{1pt}

\textbf{Objetivo:} Comprender las diferencias y similitudes entre los sistemas
de medida internacional y de prefijo binario.

\textbf{Recursos web:}

\vspace{-2\topsep}
\begin{itemize}

    \itemsep2pt \parskip0pt \parsep0pt

    \item Wikipedia: \emph{Prefijo binario}.
        \url{http://es.wikipedia.org/wiki/Prefijo_binario}

    \item Wikipedia: \emph{Prefijos del sistema internacional}.
        \url{https://es.wikipedia.org/wiki/Prefijos_del_Sistema_Internacional}

\end{itemize}

\textbf{Lectura obligatoria:}

\vspace{-2\topsep}
\begin{itemize}

    \itemsep2pt \parskip0pt \parsep0pt

    \item Apuntes de cátedra. Capitulo 2: Unidades de Información. Disponible
        en: \url{https://egrosclaude.github.io/IC/IC-notes.pdf}

\end{itemize}

\section{Unidades de información}

\begin{enumerate}

    \item Utilice la tabla \ref{tablaPrefijoSI} con prefijos del Sistema
        Internacional \emph{(SI)} de la página \pageref{tablaPrefijoSI} para
        expresar la distancia de 300 Megámetros \emph{(Mm)} en:

    \begin{enumerate*}[itemjoin=\hspace{2em}]

        \item Kilómetros~\emph{(km)}

        \item Metros~\emph{(m)}

        \item Milímetros~\emph{(mm)}

        \item Micrómetros~\emph{(µm)}

        \item Nanómetros~\emph{(nm)}

    \end{enumerate*}

    \item Exprese el tiempo de un año (considerando que un año tiene 365 días)
        en:

    \begin{enumerate*}[itemjoin=\hspace{2em}]

        \item Horas

        \item Minutos

        \item Segundos

        \item Milisegundos

        \item Microsegundos
            \hspace{2em} % Relleno para que el siguiente item bien acomodado

        \item Nanosegundos

    \end{enumerate*}

    \item Las siguientes cantidades son dadas en \textbf{prefijos
        binarios}(\url{http://es.wikipedia.org/wiki/Prefijo_binario}), exprese
        su cantidad equivalente en bits y bytes (Utilice la tabla
        \ref{tablaComparacionPrefijos} de la página
        \pageref{tablaComparacionPrefijos}).

    \begin{enumerate*}[itemjoin=\hspace{2em}]

        \item 64\emph{KiB}

        \item 4\emph{GiB}

        \item 2\emph{TiB}

    \end{enumerate*}

    \item Las siguientes cantidades son dadas en \textbf{prefijos
        decimales}(\url{https://es.wikipedia.org/wiki/Prefijos_del_Sistema_Internacional}),
        exprese su cantidad equivalente en bytes y bits  (Utilice la tabla
        \ref{tablaComparacionPrefijos} de la página
        \pageref{tablaComparacionPrefijos}).

    \begin{enumerate*}[itemjoin=\hspace{2em}]

        \item 64\emph{KB}

        \item 4\emph{GB}

        \item 2\emph{TB}

    \end{enumerate*}

        \item Al comprar un dispositivo o medio de almacenamiento secundario
            (disco rígido, pendrive, DVD) normalmente encontramos que el
            fabricante especifica la capacidad empleando prefijos decimales
            (\emph{KB}, \emph{MB}, \emph{TB}, etc.). Sin embargo,
            generalmente, un explorador de archivos muestra este dato
            utilizando prefijos binarios (\emph{KiB}, \emph{MiB}, \emph{TiB},
            etc.). Indique la capacidad que mostraría el explorador de
            archivos para dispositivos o medios de:

        \begin{enumerate*}[itemjoin=\hspace{2em}]

            \item 3\emph{MB}

            \item 4.7\emph{GB}

            \item 5\emph{TB}

        \end{enumerate*}

    \item Necesito comprar un pendrive para guardar 1990 fotos de 2 \emph{MiB}
        cada una.

        \begin{enumerate}

            \item ¿Cuántos \emph{GiB} de almacenamiento se necesitan?

            \item En un comercio hay pendrives disponibles de 2\emph{GB},
                4\emph{GB}, 8\emph{GB} y 16\emph{GB}, ¿cuál debería elegir de
                tal manera que pueda guardar todas las fotos y sobre el menor
                espacio posible?

        \end{enumerate}

    \item Aunque ambas nomenclaturas están estandarizadas, es normal que se
        utilice únicamente la de prefijos decimales, y debamos interpretar si
            se refiere a prefijo decimal o binario según el contexto.
            Supongamos que alguien envió un email diciendo: \emph{``He
            comprado un pendrive de 1GB y le he copiado una foto de 5MB"}.

        \begin{enumerate}

            \item ¿Cuántos bytes de capacidad tiene el pendrive?

            \item ¿Cuántos bytes tiene la foto?

        \end{enumerate}

\end{enumerate}

\clearpage

\begin{table}[h]

    \centering

    \caption{Prefijos del Sistema Internacional}
    \label{tablaPrefijoSI}

    \begin{tabular}{ | c | c | c | }
        \hline
        Prefijo & Símbolo & Equivalencia a la unidad \\
        \hline
        \textbf{T} & tera & $10^{12}=1\,000^{4}$ \\
        \hline
        \textbf{G} & giga & $10^{9}=1\,000^{3}$ \\
        \hline
        \textbf{M} & mega & $10^{6}=1\,000^{2}$ \\
        \hline
        \textbf{K} & kilo & $10^{3}=1\,000^{1}$ \\
        \hline
        \multicolumn{2}{|c|}{\emph{sin prefijo}} & $10^{0}=1\,000^{0}=1$ \\
        \hline
        \textbf{m} & mili & $10^{-3}=1\,000^{-1}$ \\
        \hline
        \textbf{µ} & micro & $10^{-6}=1\,000^{-2}$ \\
        \hline
        \textbf{n} & nano & $10^{-9}=1\,000^{-3}$ \\
        \hline
    \end{tabular}

    \vspace{2\topsep}
    Ejemplo:
    \vspace{-1\topsep}
    \begin{itemize}

        \itemsep2pt \parskip0pt \parsep0pt
        \item   Un \emph{kilo}gramo son $10^{3}$ gramos.
        \item   Un \emph{nano}litro son $10^{-9}$ litros.

    \end{itemize}

\end{table}

\begin{table}[h]

    \centering

    \caption{Prefijos decimales y binarios}
    \label{tablaComparacionPrefijos}

    \begin{tabular}{ | c | c | }
        \hline
        \textbf{Prefijos decimales} &
        \textbf{prefijos binarios}\\
        \hline
        \hline
        $kilobyte(\emph{KB})=10^{3}\emph{bytes}=1\,000^{1}$\emph{bytes} &
        $kibibyte(\emph{KiB})=2^{10}\emph{bytes}=1\,024^{1}$\emph{bytes} \\
        \hline
        $megabyte(\emph{MB})=10^{6}\emph{bytes}=1\,000^{2}$\emph{bytes} &
        $mebibyte(\emph{MiB})=2^{20}\emph{bytes}=1\,024^{2}$\emph{bytes} \\
        \hline
        $gigabyte(\emph{GB})=10^{9}\emph{bytes}=1\,000^{3}$\emph{bytes} &
        $gibibyte(\emph{GiB})=2^{30}=1\,024^{3}$\emph{bytes} \\
        \hline
        $terabyte(\emph{TB})=10^{12\emph{bytes}}=1\,000^{4}$\emph{bytes} &
        $tebibyte(\emph{TiB})=2^{40}\emph{bytes}=1\,024^{4}$\emph{bytes} \\
        \hline
        $petabyte(\emph{PB})=10^{15}\emph{bytes}=1\,000^{5}$\emph{bytes} &
        $pebibyte(\emph{PiB})=2^{50}\emph{bytes}=1\,024^{5}$\emph{bytes} \\
        \hline
        $exabyte(\emph{EB})=10^{18}\emph{bytes}=1\,000^{6}$\emph{bytes} &
        $exbibyte(\emph{EiB})=2^{60}\emph{bytes}=1\,024^{6}$\emph{bytes} \\
        \hline
        $zettabyte(\emph{ZB})=10^{21}\emph{bytes}=1\,000^{7}$\emph{bytes} &
        $zebibyte(\emph{ZiB})=2^{70}\emph{bytes}=1\,024^{7}$\emph{bytes} \\
        \hline
        $yottabyte(\emph{YB})=10^{24}\emph{bytes}=1\,000^{8}$\emph{bytes} &
        $yobibyte(\emph{YiB})=2^{80}\emph{bytes}=1\,024^{8}$\emph{bytes} \\
        \hline
    \end{tabular}

    \vspace{2\topsep}
    Ejemplo:
    \vspace{-1\topsep}
    \begin{itemize}

        \itemsep2pt \parskip0pt \parsep0pt
        \item   Un \emph{kilo}byte son $1\,000^{1}$ bytes.
        \item   Un \emph{mebi}byte son $2^{20}$ bytes.

    \end{itemize}

\end{table}

\end{document}
