%%% LaTeX Template: Article/Thesis/etc. with colored headings and special fonts
%%%
%%% Source: http://www.howtotex.com/
% vim: set spell spelllang=es syntax=tex :

\documentclass[12pt]{article}
\usepackage[T1]{fontenc}
\usepackage{styles/apuntes-estilo}
\usepackage{styles/egyptian}
\usepackage{fancyhdr,lastpage}
\usepackage{hyperref}
\usepackage[inline]{enumitem}
\usepackage{pmboxdraw} % Para poder usar la salida del comando tree
\usepackage{graphicx}

\def\maketitle{

\makeatletter{
    \color{blue} \centering \huge \sc
    \textbf{
        Examen final regular
    }\\
    \par
}
\makeatother

\makeatletter
% vim: set spell spelllang=es syntax=tex :
 {\centering \small 
    Introducción a la computación\\
    Departamento de Ingeniería de Computadoras \\
    Facultad de Informática - Universidad Nacional del Comahue \\
    \vspace{20pt} }
\makeatother

\vspace{-2.5cm}
\mbox{\hspace{-1cm}\includegraphics[width=3cm,height=3cm]{logos/uncoma.pdf}\hspace{13cm}
    \includegraphics[width=2.9cm,height=2.9cm]{logos/fai.pdf}}



}

% Custom headers and footers
\fancyhf{} % clear all header and footer fields
\fancypagestyle{plain}{\fancyhf{}}
\pagestyle{fancy}
\lhead{\footnotesize Primer examen de promoción}
\rhead{\footnotesize \thepage\ }

\def\ti#1#2{\texttt{#1} & #2 \\ }

\begin{document}

\begin{minipage}[c]{\linewidth}
\thispagestyle{empty}
\maketitle
\setlength{\parindent}{1pt}

\begin{enumerate}[topsep=4pt,itemsep=2pt,partopsep=2pt, parsep=2pt]

    \item \textbf{Representación de datos numéricos}
        \small{(tiempo estimado: 15 minutos)}
        ¿Qué significa que una operación aritmética tuvo ``overflow''? ¿Es
        importante detectarlo? Justifique.

    \item \textbf{Organización de Computadoras}
        \small{(tiempo estimado: 25 minutos)}
        Describa características y funcionalidad (lo mas completo que pueda)
        de cada componente de una computadora (son 4). Aclaración: En este
        ejercicio, cuando decimos ``computadora'' nos referimos a una
        computadora general (llamadas también de arquitectura de von Neumman).
        Puede ejemplificar con la \texttt{MCBE} u otra.

    \item \textbf{MCBE}
        \small{(tiempo estimado: 15 minutos)} ¿Cuál es el tamaño de una
        instrucción de la máquina \texttt{MCBE}? Si ese tamaño se amplía ¿para
        qué puede ser útil? (presente al menos dos opciones de utilidad).

    \item \textbf{El Software}
        \small{(tiempo estimado: 20 minutos)} ¿Cómo se denomina el proceso de
        conversión de código fuente a programa ejecutable? Explique el
        proceso.

    \item \textbf{Sistemas Operativos}
        \small{(tiempo estimado: 20 minutos)} Explique qué entiende por
        ``llamada al sistema (\texttt{system call})'' en un Sistema Operativo
        ¿Para qué se utilizan? Explique brevemente su funcionamiento.

\end{enumerate}
\end{minipage}

%\pagebreak
\vspace{5em}

\begin{minipage}[c]{\linewidth}
\thispagestyle{empty}
\maketitle
\setlength{\parindent}{1pt}

\begin{enumerate}[topsep=4pt,itemsep=2pt,partopsep=2pt, parsep=2pt]

    \item \textbf{Sistemas Operativos}
        \small{(tiempo estimado: 20 minutos)} Explique qué entiende por
        ``llamada al sistema (\texttt{system call})'' en un Sistema Operativo
        ¿Para qué se utilizan? Explique brevemente su funcionamiento.

    \item \textbf{El Software}
        \small{(tiempo estimado: 20 minutos)} ¿Cómo se denomina el proceso de
        conversión de código fuente a programa ejecutable? Explique el
        proceso.

    \item \textbf{MCBE}
        \small{(tiempo estimado: 15 minutos)} ¿Cuál es el tamaño de una
        instrucción de la máquina \texttt{MCBE}? Si ese tamaño se amplía ¿para
        qué puede ser útil? (presente al menos dos opciones de utilidad).

    \item \textbf{Organización de Computadoras}
        \small{(tiempo estimado: 25 minutos)}
        Describa características y funcionalidad (lo mas completo que pueda)
        de cada componente de una computadora (son 4). Aclaración: En este
        ejercicio, cuando decimos ``computadora'' nos referimos a una
        computadora general (llamadas también de arquitectura de von Neumman).
        Puede ejemplificar con la \texttt{MCBE} u otra.

    \item \textbf{Representación de datos numéricos}
        \small{(tiempo estimado: 15 minutos)}
        ¿Qué significa que una operación aritmética tuvo ``overflow''? ¿Es
        importante detectarlo? Justifique.

\end{enumerate}
\end{minipage}

\end{document}
