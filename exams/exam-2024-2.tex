%%% lAtEx tEMPLATE: Article/Thesis/etc. with colored headings and special fonts
%%%
%%% Source: http://www.howtotex.com/
% vim: set spell spelllang=es syntax=tex :

\documentclass[12pt]{article}
\usepackage[T1]{fontenc}
\usepackage{styles/apuntes-estilo}
\usepackage{styles/egyptian}
\usepackage{fancyhdr,lastpage}
\usepackage{hyperref}
\usepackage[inline]{enumitem}
\usepackage{pmboxdraw} % Para poder usar la salida del comando tree

\def\maketitle{

\makeatletter{
    \color{blue} \centering \huge \sc
    \textbf{
        Segundo examen parcial\\
        \small \vspace*{-8pt} \color{black}
        Tema: 3
        \vspace*{8pt}
    }\\
    \par
}
\makeatother

\makeatletter
% vim: set spell spelllang=es syntax=tex :
 {\centering \small 
    Introducción a la computación\\
    Departamento de Ingeniería de Computadoras \\
    Facultad de Informática - Universidad Nacional del Comahue \\
    \vspace{20pt} }
\makeatother

\vspace{-2.5cm}
\mbox{\hspace{-1cm}\includegraphics[width=3cm,height=3cm]{logos/uncoma.pdf}\hspace{13cm}
    \includegraphics[width=2.9cm,height=2.9cm]{logos/fai.pdf}}



}

% Custom headers and footers
\fancyhf{} % clear all header and footer fields
\fancypagestyle{plain}{\fancyhf{}}
\pagestyle{fancy}
\lhead{\footnotesize Segundo examen parcial}
\rhead{\footnotesize \thepage\ }

\def\ti#1#2{\texttt{#1} & #2 \\ }

\begin{document}

\thispagestyle{empty}
\maketitle
\setlength{\parindent}{1pt}


\begin{enumerate}

    \item Si se utiliza un esquema de compresión \emph{RLE(Run Lenght
        Encoding)} utilizando una codificación ``\emph{cantidad/color}'' de
        \textbf{2+2} (dos bits para la cantidad y dos bits para el color),
        donde la paleta de colores es 00 = blanco, 01 = amarillo, 10 = naranja
        y 11 = negro.

        \begin{enumerate}

            \item ¿Cuál es la imagen codificada en la siguiente secuencia de
                bytes representada en hexadecimal: \textbf{04 05 02 89 89 87
                46 4B E7}?

            \item ¿Cuántos bytes ocupa la imagen sin comprimir? \textbf{No
                olvide incluir la cabecera.}

        \end{enumerate}

    \item Dado el siguiente programa:

        \begin{verbatim}
     0: START:  LD  I
     1:         ADD UNO
     2:         ST  I
     3:         ST  OUT
     4:         SUB MAX
     5:         JZ  FIN
     6:         JMP START
     7: FIN:    HLT
     8: I:      0
     9: MAX:    2
    10: UNO:    1
        \end{verbatim}

        \begin{enumerate}

            \item ¿Cuál es la salida del programa?

            \item ¿Cuál es el equivalente en código maquina binario de las
                instrucciones de las direcciones \textbf{0}, \textbf{5} y
                \textbf{6}?

        \end{enumerate}

    \item Dado el siguiente programa en binario:

        \begin{verbatim}
     0: 01001010
     1: 11100111
     2: 10101100
     3: 01101010
     4: 01001011
     5: 01111111
     6: 11011010
     7: 00100000
     8: 01111111
     9: 11011110
    10: 00000010
    11: 00000101
    12: 00000010
        \end{verbatim}

        ¿Cuál es la salida del programa?

    \item Sea un formato de instrucción que utiliza 5 bits para el código de
        operación y 11 para el operando que indica el número de la celda de
        memoria: ¿Cual es el máximo de instrucciones y celdas de memoria que
        se pueden acceder con este formato de instrucción?

    \item Muestre una secuencia de comandos que permita crear la estructura de
        directorios mostrada en el anexo. Asuma que el
        directorio raíz (``/'') ya existe.

    \item Si el usuario se encuentra posicionado en el directorio
        \emph{fuentes} del árbol de directorios mostrado en el anexo:

        \begin{enumerate}

            \item Indique una secuencia de comandos para mover el archivo
                \emph{tp02.pdf} a la carpeta \emph{tps} utilizando solo rutas
                relativas.

            \item Indique una secuencia de comandos para mover el archivo
                \emph{correcciones.txt} a la carpeta \emph{tps} utilizando
                solo rutas absolutas.

        \end{enumerate}

\end{enumerate}

\subsection*{ \large\textbf{Anexo:} }

\begin{minipage}{0.45\textwidth}
\textbf{Códigos de operación}

\begin{verbatim}
    010     LD
    011     ST
    100     ADD
    101     SUB
    110     JMP
    111     JZ
    001     HLT
    000     NOP
\end{verbatim}
\end{minipage}
~
\begin{minipage}{0.45\textwidth}
\textbf{Estructura de directorios}

\begin{verbatim}
/
├── parciales
│   ├── 2022
│   │   └── correcciones.txt
│   └── 2021
│       └── notas.tar.gz
└── tps
    ├── tp01.pdf
    └── fuentes
        ├── tp01.tex
        ├── tp02.tex
        └── tp02.pdf
\end{verbatim}

\end{minipage}

\end{document}
