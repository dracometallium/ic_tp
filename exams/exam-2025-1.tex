%%% LaTeX Template: Article/Thesis/etc. with colored headings and special fonts
%%%
%%% Source: http://www.howtotex.com/
% vim: set spell spelllang=es syntax=tex :

\documentclass[12pt]{article}
\usepackage[T1]{fontenc}
\usepackage{styles/apuntes-estilo}
\usepackage{styles/egyptian}
\usepackage{fancyhdr,lastpage}
\usepackage{hyperref}
\usepackage[inline]{enumitem}
\usepackage{pmboxdraw} % Para poder usar la salida del comando tree

\def\maketitle{

\makeatletter{
    \color{blue} \centering \huge \sc
    \textbf{Primer examen parcial}\\
    \par
}
\makeatother

\makeatletter
% vim: set spell spelllang=es syntax=tex :
 {\centering \small 
    Introducción a la computación\\
    Departamento de Ingeniería de Computadoras \\
    Facultad de Informática - Universidad Nacional del Comahue \\
    \vspace{20pt} }
\makeatother

\vspace{-2.5cm}
\mbox{\hspace{-1cm}\includegraphics[width=3cm,height=3cm]{logos/uncoma.pdf}\hspace{13cm}
    \includegraphics[width=2.9cm,height=2.9cm]{logos/fai.pdf}}



}

% Custom headers and footers
\fancyhf{} % clear all header and footer fields
\fancypagestyle{plain}{\fancyhf{}}
\pagestyle{fancy}
\lhead{\footnotesize Examen final libre}
\rhead{\footnotesize \thepage\ }

\def\ti#1#2{\texttt{#1} & #2 \\ }

\begin{document}

\thispagestyle{empty}
\maketitle
\setlength{\parindent}{1pt}

\begin{enumerate}

    \item Si se tiene una imagen con 200 colores distintos ¿Cuál es la
        cantidad mínima de bits necesarios para la profundidad de color, de
        manera de poder representar la imagen sin pérdida de colores

    \item ¿Cuál es la representación de $42_{7}$ en las siguientes bases?

        \begin{enumerate}

            \item Base 3.

            \item Base 2.

            \item Base 16.

        \end{enumerate}

    \item ¿Cuál es rango de representación de los sistemas \emph{complemento a
    2}, \emph{signo magnitud} y \emph{sin signo} con 9 bits?

    \item Indique la representación de los siguientes números en
    \emph{complemento a 2} con 8 bits:

        \begin{enumerate}

            \item 16.
            \item -16.
            \item 100.
            \item -100.
            \item -128.

        \end{enumerate}

    \item Indique la representación de los siguientes números en
    \emph{signo magnitud} con 8 bits:

        \begin{enumerate}

            \item 16.
            \item -16.
            \item 100.
            \item -100.
            \item -128.

        \end{enumerate}

    \item Resuelva cada una de las siguientes operaciones de números
    representados en \emph{complemento a 2} e indique en cuales se produce
    \emph{overflow}:

        \begin{enumerate}

            \item $1010\,1010 + 1111 1111$
            \item $1010\,1010 - 1111 1111$
            \item $0101\,1100 + 0010 0010$
            \item $0101\,1100 - 0010 0010$
            \item $1001\,1100 - 0010 0010$

        \end{enumerate}

    \item ¿Cual es la representación en \emph{punto fijo} y \emph{complemento
        a 2} de los siguientes números si se utilizan 4 bits para la parte
        entera y 4 para la parte fraccionaria? Indicar cual es el error de
        representación (la diferencia entre el valor original y el
        almacenado).

        \begin{enumerate}

            \item $7.4$

            \item $-7.4$

        \end{enumerate}

    \item ¿Que mensaje esta codificado en la siguiente secuencia de bytes
        expresados en hexadecimal: \textbf{46 69 6E 61 6C 20 64 65  20 49 43 2E}?
        \emph{Respetar mayúsculas y minúsculas.}

    \item Si se tienen 256 imágenes de 8\textbf{MB} cada una:

        \begin{enumerate}

            \item ¿Cuántos \emph{GB} ocupara el total de los archivos?

            \item Si se tiene un dispositivo de almacenamiento de
            2\textbf{GiB} ¿Se podrá almacenar el total de los archivos?

        \end{enumerate}

    \item Sea \textbf{0xC5} un número en base 16 (hexadecimal) que
        representa a un número real expresado en el sistema de \emph{Punto
        Flotante MiniFloat}.

        \begin{enumerate}

            \item Indique cuál es el valor en base 10 de dicho número real.

            \item Indique cuál es el error de representación.

        \end{enumerate}


\end{enumerate}

\subsection*{ \large\textbf{Anexo:} }

\textbf{Tabla \emph{ASCII}:}

\begin{verbatim}
Dec Hex    Dec Hex    Dec Hex  Dec Hex  Dec Hex  Dec Hex   Dec Hex   Dec Hex
  0 00 NUL  16 10 DLE  32 20    48 30 0  64 40 @  80 50 P   96 60 `  112 70 p
  1 01 SOH  17 11 DC1  33 21 !  49 31 1  65 41 A  81 51 Q   97 61 a  113 71 q
  2 02 STX  18 12 DC2  34 22 "  50 32 2  66 42 B  82 52 R   98 62 b  114 72 r
  3 03 ETX  19 13 DC3  35 23 #  51 33 3  67 43 C  83 53 S   99 63 c  115 73 s
  4 04 EOT  20 14 DC4  36 24 $  52 34 4  68 44 D  84 54 T  100 64 d  116 74 t
  5 05 ENQ  21 15 NAK  37 25 %  53 35 5  69 45 E  85 55 U  101 65 e  117 75 u
  6 06 ACK  22 16 SYN  38 26 &  54 36 6  70 46 F  86 56 V  102 66 f  118 76 v
  7 07 BEL  23 17 ETB  39 27 '  55 37 7  71 47 G  87 57 W  103 67 g  119 77 w
  8 08 BS   24 18 CAN  40 28 (  56 38 8  72 48 H  88 58 X  104 68 h  120 78 x
  9 09 HT   25 19 EM   41 29 )  57 39 9  73 49 I  89 59 Y  105 69 i  121 79 y
 10 0A LF   26 1A SUB  42 2A *  58 3A :  74 4A J  90 5A Z  106 6A j  122 7A z
 11 0B VT   27 1B ESC  43 2B +  59 3B ;  75 4B K  91 5B [  107 6B k  123 7B {
 12 0C FF   28 1C FS   44 2C ,  60 3C <  76 4C L  92 5C \  108 6C l  124 7C |
 13 0D CR   29 1D GS   45 2D -  61 3D =  77 4D M  93 5D ]  109 6D m  125 7D }
 14 0E SO   30 1E RS   46 2E .  62 3E >  78 4E N  94 5E ^  110 6E n  126 7E ~
 15 0F SI   31 1F US   47 2F /  63 3F ?  79 4F O  95 5F _  111 6F o  127 7F DEL
\end{verbatim}

\end{document}
