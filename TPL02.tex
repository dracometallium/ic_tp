%%% LaTeX Template: Article/Thesis/etc. with colored headings and special fonts
%%%
%%% Source: http://www.howtotex.com/
% vim: set spell spelllang=es syntax=tex :

\documentclass[12pt]{article}
\usepackage{styles/apuntes-estilo}
\usepackage{fancyhdr,lastpage}
\usepackage{hyperref}
\usepackage[inline]{enumitem}
\usepackage{xurl}
\usepackage{upquote}

\newcommand{\cw}[1]{\texttt{\textcolor{blue}{#1}}}

\newcommand{\bash}{\textbf{\texttt{BASH}}}

\def\maketitle{

\makeatletter{
    \color{blue} \centering \huge \sc
    \textbf{
        Trabajo práctico de Laboratorio 02\\
        \large \vspace*{-8pt} \color{black}
        Introducción al shell del sistema GNU/LINUX
        \vspace*{8pt}
    }
    \par
}

\makeatother

\makeatletter
% vim: set spell spelllang=es syntax=tex :
 {\centering \small 
    Introducción a la computación\\
    Departamento de Ingeniería de Computadoras \\
    Facultad de Informática - Universidad Nacional del Comahue \\
    \vspace{20pt} }
\makeatother

\vspace{-2.5cm}
\mbox{\hspace{-1cm}\includegraphics[width=3cm,height=3cm]{logos/uncoma.pdf}\hspace{13cm}
    \includegraphics[width=2.9cm,height=2.9cm]{logos/fai.pdf}}



}

% Custom headers and footers
\fancyhf{} % clear all header and footer fields
\fancypagestyle{plain}{\fancyhf{}}
\pagestyle{fancy}
\lhead{\footnotesize Trabajo práctico de Laboratorio 02 - Introducción al shell del sistema GNU/LINUX}
\rhead{\footnotesize \thepage\ }

\def\ti#1#2{\texttt{#1} & #2 \\ }

\begin{document}

\thispagestyle{empty}
\maketitle
\setlength{\parindent}{1pt}

\textbf{Lectura obligatoria:}

\vspace{-2\topsep}
\begin{itemize}

    \itemsep2pt \parskip0pt \parsep0pt

    \item Apunte del shell de Linux:
        \url{http://pedco.uncoma.edu.ar/mod/resource/view.php?id=207175}

    \item Apunte introductorio a \bash:
        \url{http://pedco.uncoma.edu.ar/mod/resource/view.php?id=244968}

    \item Linux Man Pages Online: \url{https://linux.die.net/man/}

\end{itemize}

\vspace{-2\topsep}

\section{Administración de procesos}

\begin{enumerate}

    \item Cree un archivo con nombre \cw{dance.sh} con el siguiente contenido.
        Verifique que ha copiado correctamente cada linea.

        \begin{verbatim}
#!/bin/sh

while true; do
    printf "\t(>'-')>\r"
    sleep 0.5
    printf "\t<('-'<)\r"
    sleep 0.5
done
        \end{verbatim}

    \item Agregue permisos de ejecución utilizando el comando \cw{chmod +x
        dance.sh}.

    \item Ejecute el programa ingresando en la terminal \cw{./dance.sh}.

    \item Abra otra ventana y utilizando el comando \cw{ps a} encuentre el
        \cw{PID} del proceso del programa \cw{dance.sh}. Termine el programa
        utilizando el comando \cw{kill \emph{PID\_DEL\_PROCESO}}.

    \item Ejecute el comando \cw{file} sobre el archivo \cw{dance.sh} y
        \cw{/bin/ls} ¿Cuál es la función del comando \cw{file}?

    %\item ¿En qué tipo de lenguaje esta escrito el programa \cw{dance.sh}:
        %compilado o interpretado? ¿Y el comando \cw{ls}?

\end{enumerate}

\section{Lenguajes compilados e interpretados}

\subsection{Lenguajes interpretados}

\begin{enumerate}[resume]

    \item Cree un archivo con nombre \cw{calcular.sh} con el siguiente
        contenido. Verifique que ha copiado correctamente cada linea.

        \begin{verbatim}
#!/bin/sh
max=${1:-10}
num=0
for i in $(seq 1 "$max");do
    num=$((num + 3))
done
echo "El resultado de sumar 3 unas $max veces es: $num"
        \end{verbatim}

    \item Agregue permisos de ejecución utilizando el comando \cw{chmod +x
        calcular.sh}. Mida el tiempo de ejecución para $100$, $100\,000$ y
        $1\,000\,000$ utilizando el comando \cw{time ./calcular.sh \$NUM}
        (remplace \cw{\$NUM} por cada uno de los valores).

\end{enumerate}

\subsection{Lenguajes compilados}

\begin{enumerate}[resume]

    \item Cree un archivo con nombre \cw{calcular.c} con el siguiente
        contenido. Verifique que ha copiado correctamente cada linea.

        \begin{verbatim}
#include<stdio.h>
#include<stdlib.h>
#include<stdint.h>

int main(int carg, char **varg)
{
    uint64_t max = 10, i, num = 0;
    if (carg > 1) {
        max = atoi(varg[1]);
    }
    for (i = 0; i < max; i++) {
        num = num + 3;
    }
    printf("El resultado de sumar 3 unas %d veces es: %d\n", max, num);
    return 0;
}
        \end{verbatim}

    \item Cree un archivo ensamblador optimizado con el comando \cw{gcc
        calcular.c -S -O3} (\cw{-O} es la letra ``O'' mayuscula). Inspeccione
        el contenido del archivo \cw{calcular.s}.

    \item Cree un archivo ejecutable con el comando \cw{gcc calcular.s -o
        calcular} (\cw{-o} es la letra ``o'' minuscula). Ejecute el comando
        \cw{file calcular} ¿Qué información retorna el comando?

    \item Mida el tiempo de ejecución para $100$, $100\,000$ y $1\,000\,000$
        utilizando el comando \cw{time ./calcular \$NUM} (remplace \cw{\$NUM}
        por cada uno delos valores).

\end{enumerate}

\subsection{Comparando resultados}

\begin{enumerate}[resume]

    \item ¿Cuál de los dos ejecutables fue más rápido?

    \item Modifique ambos programas para que sumen el número 42 en lugar de 3.
        Verifique que el cambio se realizo con exito sumando 100 veces.

    \item ¿Cuál considera que fue más simple de modificar?

    \item Indique en sus palabras cuales son las ventajas y desventajas de los
        lenguajes compilados y los interpretados.

\end{enumerate}

\end{document}
