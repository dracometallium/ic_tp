%%% LaTeX Template: Article/Thesis/etc. with colored headings and special fonts
%%%
%%% Source: http://www.howtotex.com/
% vim: set spell spelllang=es syntax=tex :

\documentclass[12pt]{article}
\usepackage{styles/apuntes-estilo}
\usepackage{fancyhdr,lastpage}
\usepackage{hyperref}
\usepackage[inline]{enumitem}
\usepackage{xurl}
\usepackage{upquote}

\def\maketitle{

\makeatletter{
    \color{blue} \centering \huge \sc
    \textbf{
        Trabajo práctico de laboratorio N° 11\\
        \large \vspace*{-8pt} \color{black}
        Administración de procesos en sistema GNU/LINUX
        \vspace*{8pt}
    }\\
    \small Fecha de finalización: 10 de Junio
    \par
}

\makeatother

\makeatletter
% vim: set spell spelllang=es syntax=tex :
 {\centering \small 
    Introducción a la computación\\
    Departamento de Ingeniería de Computadoras \\
    Facultad de Informática - Universidad Nacional del Comahue \\
    \vspace{20pt} }
\makeatother

\vspace{-2.5cm}
\mbox{\hspace{-1cm}\includegraphics[width=3cm,height=3cm]{logos/uncoma.pdf}\hspace{13cm}
    \includegraphics[width=2.9cm,height=2.9cm]{logos/fai.pdf}}



}

% Custom headers and footers
\fancyhf{} % clear all header and footer fields
\fancypagestyle{plain}{\fancyhf{}}
\pagestyle{fancy}
\lhead{\footnotesize Práctico N° 11 - Administración de procesos en sistema GNU/LINUX }
\rhead{\footnotesize \thepage\ }

\def\ti#1#2{\texttt{#1} & #2 \\ }

\newcommand{\bash}{\textbf{\emph{BASH}}}

\begin{document}

\thispagestyle{empty}
\maketitle
\setlength{\parindent}{1pt}

\textbf{Lectura obligatoria:}

\vspace{-2\topsep}
\begin{itemize}

    \itemsep2pt \parskip0pt \parsep0pt

    \item Apunte del shell de Linux:
        \url{http://pedco.uncoma.edu.ar/mod/resource/view.php?id=207175}

    \item Apunte introductorio a \bash:
        \url{http://pedco.uncoma.edu.ar/mod/resource/view.php?id=244968}

    \item Linux Man Pages Online: \url{https://linux.die.net/man/}

\end{itemize}

A continuación, se realizarán una serie de ejercicios para los cuales
necesitará acceso a una computadora con el sistema operativo Linux (o una
Máquina Virtual corriendo Linux) y las siguientes aplicaciones:

\vspace{-2\topsep}
\begin{itemize}

    \itemsep2pt \parskip0pt \parsep0pt

    \item   Intérprete de comandos \bash y demás utilidades que se encuentran
        en la mayoría de las distribuciones de Linux.

    \item   Un editor de texto en la interfaz de línea de comando o gráfica.

\end{itemize}

\section*{Administración de procesos}

\begin{enumerate}

    \item Cree un archivo con nombre \emph{dance.sh} con el siguiente
        contenido:

        \begin{verbatim}
#!/bin/sh

while true; do
    printf "\t(>'-')>\r"
    sleep 0.5
    printf "\t<('-'<)\r"
    sleep 0.5
    printf "\t^('-')^\r"
    sleep 0.5
    printf "\t<('-'<)\r"
    sleep 0.5
done
        \end{verbatim}

    \item Agregue permisos de ejecución utilizando el comando \emph{chmod} al
        archivo \emph{dance.sh} (puede usar el comando \emph{man chmod} para
        más información de como se utiliza el comando).

    \item Ejecute el programa ingresando en la terminal \emph{./dance.sh} ¿En
        qué tipo de lenguaje esta escrito este programa: compilado o
        interpretado?

    \item Abra otra ventana y utilizando los comandos \emph{top} o \emph{ps a}
        encuentre el \emph{PID} del proceso del programa \emph{dance.sh}.
        Termine el programa utilizando el comando \emph{kill
        \textbf{PID\_DEL\_PROCESO}}.

\end{enumerate}

\end{document}
