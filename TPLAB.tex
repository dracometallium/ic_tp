%%% LaTeX Template: Article/Thesis/etc. with colored headings and special fonts
%%%
%%% Source: http://www.howtotex.com/
% vim: set spell spelllang=es syntax=tex :

\documentclass[12pt]{article}
\usepackage{styles/apuntes-estilo}
\usepackage{styles/egyptian}
\usepackage{fancyhdr,lastpage}
\usepackage{hyperref}
\usepackage[inline]{enumitem}

\def\maketitle{

\makeatletter
{\color{bl} \centering \huge \sc \textbf{ Trabajo práctico de laboratorio N° 1\\ \large
\vspace*{-8pt} \color{black} Introducción al shell del sistema GNU/LINUX \vspace*{8pt} }\par}
\makeatother

\makeatletter
% vim: set spell spelllang=es syntax=tex :
 {\centering \small 
    Introducción a la computación\\
    Departamento de Ingeniería de Computadoras \\
    Facultad de Informática - Universidad Nacional del Comahue \\
    \vspace{20pt} }
\makeatother

\vspace{-2.5cm}
\mbox{\hspace{-1cm}\includegraphics[width=3cm,height=3cm]{logos/uncoma.pdf}\hspace{13cm}
    \includegraphics[width=2.9cm,height=2.9cm]{logos/fai.pdf}}



}

% Custom headers and footers
\fancyhf{} % clear all header and footer fields
\fancypagestyle{plain}{\fancyhf{}}
\pagestyle{fancy}
\lhead{\footnotesize Laboratorio N° 1 - Introducción al shell del sistema GNU/LINUX}
\rhead{\footnotesize \thepage\ }

\def\ti#1#2{\texttt{#1} & #2 \\ }

\newcommand{\bash}{\emph{BASH}}

\begin{document}

\thispagestyle{empty}
\maketitle
\setlength{\parindent}{1pt}

A continuación, se realizarán una serie de ejercicios para los cuales
necesitará acceso a una computadora con el sistema operativo Linux (o una
Máquina Virtual corriendo Linux) y las siguientes aplicaciones:

\begin{itemize}
        
    \item   Intérprete de comandos \bash y demás utilidades que se encuentran
        en la mayoría de las distribuciones de Linux.
        
    \item   Un editor de texto en la interfaz de línea de comando o gráfica.

\end{itemize}

El sistema operativo controla diferentes procesos de una computadora. Uno de
ellos es el intérprete de comandos o \emph{shell}. Este es un programa que
permite al usuario interactuar con el Sistema Operativo. Permite iniciar
(ejecutar) otros programas así como también tiene comandos propios que no
necesitan de otros programas, como por ejemplo comandos para movernos en la
estructura de directorios. En este práctico, utilizaremos el intérprete
\emph{BASH (Bourne-Again SHell)}, muy popular en el mundo de Linux.

Dicho programa debe ser ejecutado en lo que llamamos Terminal o Consola, que
es otro programa que nos permite interactuar mediante el teclado (ingresar
caracteres) y ver los resultados de la ejecución de otros programas en la
pantalla. Al iniciar una consola o terminal de textos, automáticamente se
inicia el \emph{shell} \bash, que es con el que el usuario realmente
interactúa (más abajo se explica esto con un ejemplo).

\section{Sistemas de archivos}

El concepto de directorios (comúnmente llamados ``carpetas'' y archivos es hoy
en día familiar a los usuarios de computadoras, en general el usuario se
maneja visualmente con el ratón y el Explorador de Archivos, utilizando estos
para acceder a los directorios, abrir archivos, ejecutar programas, etc. Podemos
pensar el sistema de archivos como una forma de organizar todo el contenido en
el dispositivo de almacenamiento. Los archivos son datos concretos, utilizan
espacio físico del dispositivo, mientras que las carpetas sirven para
organizar ``lógicamente'' las cosas, igual que una persona tiene estanterías y
cajones en un escritorio para organizar sus papeles.

La Figura 1 muestra un esquema de como se puede pensar en una estructura de directorios y archivos. A,
B, C, D, E, G y J son carpetas o directorios, F, H, I, K, L, M, N y O son archivos.

FIGURA

\end{document}
