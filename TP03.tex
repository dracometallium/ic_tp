%%% LaTeX Template: Article/Thesis/etc. with colored headings and special fonts
%%%
%%% Source: http://www.howtotex.com/
% vim: set spell spelllang=es syntax=tex :

\documentclass[12pt]{article}
\usepackage{styles/apuntes-estilo}
\usepackage{fancyhdr,lastpage}
\usepackage{hyperref}
\usepackage[inline]{enumitem}
\usepackage{xurl}

\def\maketitle{

\makeatletter{
    \color{blue} \centering \huge \sc
    \textbf{
        Trabajo práctico N° 3\\
        \large \vspace*{-8pt} \color{black}
        Representación de la información - Números enteros
        \vspace*{8pt}
    }\\
    \small Fecha de finalización: 22 de Abril de 2022
    \par
}

\makeatother

\makeatletter
% vim: set spell spelllang=es syntax=tex :
 {\centering \small 
    Introducción a la computación\\
    Departamento de Ingeniería de Computadoras \\
    Facultad de Informática - Universidad Nacional del Comahue \\
    \vspace{20pt} }
\makeatother

\vspace{-2.5cm}
\mbox{\hspace{-1cm}\includegraphics[width=3cm,height=3cm]{logos/uncoma.pdf}\hspace{13cm}
    \includegraphics[width=2.9cm,height=2.9cm]{logos/fai.pdf}}



}

% Custom headers and footers
\fancyhf{} % clear all header and footer fields
\fancypagestyle{plain}{\fancyhf{}}
\pagestyle{fancy}
\lhead{\footnotesize TP N° 3 - Representación de la información - Números
enteros}
\rhead{\footnotesize \thepage\ }

\def\ti#1#2{\texttt{#1} & #2 \\ }

\begin{document}

\thispagestyle{empty}
\maketitle
\setlength{\parindent}{1pt}

\textbf{Objetivo:} comprender la representación binaria de números enteros.

\textbf{Recursos Web:}

\vspace{-2\topsep}
\begin{itemize}

    \itemsep2pt \parskip0pt \parsep0pt

    \item Wikipedia: \emph{Complemento a 2}:
        \url{https://en.wikipedia.org/wiki/Two\%27s_complement}

\end{itemize}

\textbf{Lectura obligatoria:}

\vspace{-2\topsep}
\begin{itemize}

    \itemsep2pt \parskip0pt \parsep0pt

    \item Apuntes de cátedra. Capitulo 3: Representación de la Información.
        Disponible en: \url{https://egrosclaude.github.io/IC/IC-notes.pdf}

\end{itemize}

\begin{enumerate}

    \item Completar la siguiente tabla en el sistema binario. Para ello, 
    convierta cada número del sistema decimal al sistema binario 
    considerando que en: 
    \begin{itemize}
	  \item la columna \textbf{Sin Signo} debe utilizar la cantidad mínima de dígitos para expresar el número, y en
	  \item la columna \textbf{Sin Signo en 8 bits}, siempre se deben utilizar 8 dígitos para expresar el número.
    \end{itemize}
    
    De esta manera, por ejemplo, si el número en sistema decimal es 8, entonces en binario sin signo es 1000 y en binario sin signo en 8 bits es 00001000.

    \begin{center}
		
        \begin{tabular}[t]{|c|c|c|}

        \cline{2-3}

            \multicolumn{1}{c}{}&\multicolumn{2}{|c|}{\textbf{Sistema Binario}}\\

        \hline

            \textbf{Sistema} & \textbf{Sin Signo}& \textbf{Sin Signo}\\

            \textbf{Decimal} & ~ & \textbf{en 8 bits}\\

        \hline

            0 & \hspace{14em}~&\hspace{14em}~\\

        \hline

            40&&\\

        \hline

            80&&\\

        \hline

            147&&\\

        \hline

            255&&\\

        \hline

        \end{tabular}

    \end{center}

\item Completar la siguiente tabla con la representación en \emph{8 bits} de
    los siguientes números en \emph{Signo Magnitud} y \emph{Complemento a 2}.
    Indique con un guion aquellos casos donde no sea posible.

    \begin{center}

        \begin{tabular}[t]{|c|c|c|}

        \hline

            \textbf{Decimal} & \textbf{Signo Magnitud} & \textbf{Complemento a
            2}\\

        \hline

            3 & \hspace{14em}~&\hspace{14em}~\\

        \hline

            -3&&\\

        \hline

            66&&\\

        \hline

            -66&&\\

        \hline

            -128&&\\

        \hline

        \end{tabular}

    \end{center}

    \item Complete la siguiente tabla que representa enteros de \textbf{3
        bits}. Indique con un guion aquellos casos donde no sea posible.

        \begin{center}

            \begin{tabular}[t]{|c|c|c|c|}

            \cline{2-4}

            \multicolumn{1}{c}{}&\multicolumn{3}{|c|}{\textbf{Sistema Binario}}\\

            \hline

                \textbf{Sistema} & \textbf{Complemento a 2}& \textbf{Signo
                Magnitud} & \textbf{Sin Signo}\\

                \textbf{Decimal} & ~ & ~ &\\

            \hline

                3 & \hspace{9em}~&\hspace{9em}~&\hspace{9em}~\\

            \hline

                2&&&\\

            \hline

                1&&&\\

            \hline

                0&&&\\

            \hline

                -1&&&\\

            \hline

                -2&&&\\

            \hline

                -3&&&\\

            \hline

                -4&&&\\

            \hline

            \end{tabular}

        \end{center}

        \begin{enumerate}

            \item Una vez completada la tabla, a cada valor de la columna
                Complemento a 2 aplique la operación de complemento a 2 y
                responda: ¿Cuál es el \emph{significado aritmético} de lo que
                observamos?

            \item ¿Cuál es el rango de números representables para \textbf{3
                bits} en:

            \begin{enumerate*}[itemjoin=\hspace{2em}]

                \item Sin signo?

                \item Signo magnitud?

                \item Complemento a 2?

            \end{enumerate*}

    \end{enumerate}

    \item En cada caso indique cuál es la \textbf{\emph{fórmula}} 
    para obtener el rango de números
    representables para \textbf{n} bits si la representación se trata de:

        \begin{enumerate*}[itemjoin=\hspace{2em}]

            \item Sin signo?

            \item Signo magnitud?

            \item Complemento a 2?

        \end{enumerate*}

    \item Indicar el rango de los números representables con 4, 8, 16 y 32
        bits utilizando la notación de la siguiente tabla:

        \begin{center}

            \begin{tabular}[t]{|c|c|c|c|}

            \cline{2-4}

            \multicolumn{1}{c|}{}& \textbf{Sin Signo} &
                \textbf{Complemento a 2} & \textbf{Signo Magnitud}\\

            \hline

                4 bits & \hspace{9em}~ &\hspace{9em}~&\hspace{9em}~\\

            \hline

            8 bits&&&\\

            \hline

            16 bits&&&\\

            \hline

            32 bits&&&\\

            \hline

            \end{tabular}

        \end{center}


    \item Representar en Complemento a 2 los siguientes números enteros
        decimales. Utilizar representaciones de 8, 16 o 32 bits según sea
        necesario.

        \begin{center}

            \begin{tabular}[t]{|c|c|}

            \hline

                \textbf{Sistema} & \textbf{Complemento a 2}\\

                \textbf{Decimal} & ~ \\

            \hline

                -50 & \hspace{27em}~ \\

            \hline

                -128&\\

            \hline

                -256&\\

            \hline

                -542&\\

            \hline

                -40090&\\

            \hline

            \end{tabular}

        \end{center}

    \pagebreak %Agregado por que sino queda solo la tabla al final del TP
    \item Complete la siguiente tabla para los números hexadecimales
        representados en 8 bits de la siguiente manera: una vez haya expresado el
        	numero hexadecimal en binario, \textbf{\textit{interprete}} la secuencia 
        	de bits como un numero decimal \emph{Sin signo} y en \emph{Complemento a 2}.

        \begin{center}
            \begin{tabular}[t]{|c|c|c|c|}

            \hline

            \textbf{Hex.} & \textbf{Binario} & \textbf{Sin Signo} & \textbf{Complemento a 2}\\

            \hline

            \textbf{A3} & $1010\,0011$ & $163$ & $-93$\\

            \hline

            \textbf{2B} & \hspace{9em}~ &~&~\\

            \hline

            \textbf{9F}&&&\\

            \hline

            \textbf{F9}&&&\\

            \hline

            \end{tabular}

        \end{center}

\end{enumerate}

\end{document}
