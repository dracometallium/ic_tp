%%% LaTeX Template: Article/Thesis/etc. with colored headings and special fonts
%%%
%%% Source: http://www.howtotex.com/
% vim: set spell spelllang=es syntax=tex :

\documentclass[12pt]{article}
\usepackage{styles/apuntes-estilo}
\usepackage{fancyhdr,lastpage}
\usepackage{hyperref}
\usepackage[inline]{enumitem}
\usepackage{xurl}
\usepackage{xcolor}

\def\maketitle{

\makeatletter{
    \color{blue} \centering \huge \sc
    \textbf{
        Trabajo práctico N° 5\\
        \large \vspace*{-8pt} \color{black}
        Representación de digital de datos
        \vspace*{8pt}
    }\\
    \small Fecha de finalización: 29 de abril de 2022\\
    \color{red}
    \textbf{IMPORTANTE:}\\
    Fecha del primer examen parcial: Viernes 6 de mayo de 2022
    \par
}

\makeatother

\makeatletter
% vim: set spell spelllang=es syntax=tex :
 {\centering \small 
    Introducción a la computación\\
    Departamento de Ingeniería de Computadoras \\
    Facultad de Informática - Universidad Nacional del Comahue \\
    \vspace{20pt} }
\makeatother

\vspace{-2.5cm}
\mbox{\hspace{-1cm}\includegraphics[width=3cm,height=3cm]{logos/uncoma.pdf}\hspace{13cm}
    \includegraphics[width=2.9cm,height=2.9cm]{logos/fai.pdf}}



}

% Custom headers and footers
\fancyhf{} % clear all header and footer fields
\fancypagestyle{plain}{\fancyhf{}}
\pagestyle{fancy}
\lhead{\footnotesize TP N° 5 - Representación de la Información}
\rhead{\footnotesize \thepage\ }

\def\ti#1#2{\texttt{#1} & #2 \\ }

\begin{document}

\thispagestyle{empty}
\maketitle
\setlength{\parindent}{1pt}

\textbf{Objetivo:} comprender la representación binaria de números de (coma) punto fijo y flotante.

\textbf{Recursos web:}

\vspace{-2\topsep}
\begin{itemize}

    \itemsep2pt \parskip0pt \parsep0pt
    \item Wikipedia: \emph{IEEE coma flotante}:
        \url{http://es.wikipedia.org/wiki/IEEE_coma_flotante}

    \item Calculadora IEEE-754: \url{http://www.zator.com/Cpp/E2_2_4a1.htm}

\end{itemize}

\textbf{Lectura obligatoria:}

\vspace{-2\topsep}
\begin{itemize}

    \itemsep2pt \parskip0pt \parsep0pt

    \item Apuntes de cátedra. Capítulo 3: Representación de la Información.
        Disponible en: \url{https://egrosclaude.github.io/IC/IC-notes.pdf}

\end{itemize}

\section*{Representación de números reales}

\textbf{Nota}: La abreviatura ``Hex'' significa Hexadecimal, y el prefijo
``\textbf{0x}'' indica que un número está en hexadecimal.

\subsection*{Representación binaria de números de punto (coma) fijo}

\begin{enumerate}

    \item Representar los números reales en notación de \emph{Punto Fijo} y
        \emph{Complemento a 2}, utilizando 4 bits para la parte entera y 4
        para la parte fraccionaria:

        \begin{enumerate*}[itemjoin=\hspace{2em}]

            \item $1.75$

            \item $-1.75$

            \item $7.06$

            \item $-5.9$

        \end{enumerate*}

    \item Para cada inciso del ejercicio anterior, realice la conversión
        inversa (es decir, de Punto Fijo a expresión decimal) e indique el
        \textbf{error de precisión cometido} (la diferencia entre el número
        original y el representado).

    \item Los siguientes números están representados en \emph{Punto Fijo
        y complemento a dos en 8 bits, con cuatro bits para la parte entera y
        4 para la parte fraccionaria}. Indique a qué número decimal se corresponde:

        \begin{enumerate*}[itemjoin=\hspace{2em}]

            \item 0x41

            \item 0xF8

            \item 0xA3

        \end{enumerate*}

    \item Dados los siguientes números representados en \emph{Punto Fijo y
        Complemento a 2, con 4 bits para la parte entra y 4 bits para la parte
        fraccionaria}, efectuar las siguientes sumas y determinar cuales de ellas
        producen \emph{overflow}:

        \begin{enumerate*}[itemjoin=\hspace{2em}]

            \item $1000.1010 + 1100.0110$

            \item $0001.0000 + 1000.0001$

            \item $0111.1100 + 0111.0010$

        \end{enumerate*}

\end{enumerate}

\subsection*{Representación binaria de números de punto (coma) flotante}

\begin{enumerate}[resume]

    \item Los siguientes números están representados en \emph{Punto Flotante
        IEEE-754 de precisión simple (32 bits)}. Indique a qué número decimal
        se corresponde:

        \begin{enumerate*}[itemjoin=\hspace{2em}]

            \item 0x41700000

            \item 0x42CD8000

            \item 0x42008000

        \end{enumerate*}

    \item Convertir del sistema decimal a la notación \emph{Punto Flotante
        IEEE-754 de precisión simple (32 bits)} y mostrar el resultado final
        en notación hexadecimal:

        \begin{enumerate*}[itemjoin=\hspace{2em}]

            \item $1.75$

            \item $-0.0625$

            \item $0.3$

            \item $-5.9$

            \item $0$

            \item $-infinito$

        \end{enumerate*}

    \item Para cada inciso del ejercicio anterior, realice la conversión
        inversa (es decir, de Punto Flotante a expresión decimal) e indique el
        \textbf{error de precisión cometido}.

    \item Calcular el rango de los números reales representables con el
        formato \emph{IEEE-754 de precisión simple}.

\end{enumerate}

\end{document}
