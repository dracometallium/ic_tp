%%% LaTeX Template: Article/Thesis/etc. with colored headings and special fonts
%%%
%%% Source: http://www.howtotex.com/
% vim: set spell spelllang=es syntax=tex :

\documentclass[12pt]{article}
\usepackage{styles/apuntes-estilo}
\usepackage{styles/egyptian}
\usepackage{fancyhdr,lastpage}
\usepackage{hyperref}
\usepackage[inline]{enumitem}
\usepackage{xurl}

\def\maketitle{

\makeatletter{
    \color{blue} \centering \huge \sc
    \textbf{
        Trabajo práctico N° 1\\
        \large \vspace*{-8pt} \color{black}
        Sistemas de numeración y unidades de información
        \vspace*{8pt}
    }\\
    \small Fecha de finalización: 31 de marzo
    \par
}

\makeatother

\makeatletter
% vim: set spell spelllang=es syntax=tex :
 {\centering \small 
    Introducción a la computación\\
    Departamento de Ingeniería de Computadoras \\
    Facultad de Informática - Universidad Nacional del Comahue \\
    \vspace{20pt} }
\makeatother

\vspace{-2.5cm}
\mbox{\hspace{-1cm}\includegraphics[width=3cm,height=3cm]{logos/uncoma.pdf}\hspace{13cm}
    \includegraphics[width=2.9cm,height=2.9cm]{logos/fai.pdf}}



}

% Custom headers and footers
\fancyhf{} % clear all header and footer fields
\fancypagestyle{plain}{\fancyhf{}}
\pagestyle{fancy}
\lhead{\footnotesize TP N° 1 - Sistemas de numeración y unidades de información}
\rhead{\footnotesize \thepage\ }

\def\ti#1#2{\texttt{#1} & #2 \\ }

\begin{document}

\thispagestyle{empty}
\maketitle
\setlength{\parindent}{1pt}

\textbf{Objetivo:} Comprender el sistema de numeración posicional, conversión
entre sistemas de distintas bases, y las unidades de información.

\textbf{Recursos bibliográfico:}

\vspace{-2\topsep}
\begin{itemize}

    \itemsep2pt \parskip0pt \parsep0pt

    \item Wikipedia: \emph{Prefijo binario}.
        \url{http://es.wikipedia.org/wiki/Prefijo_binario}

    \item Wikipedia: \emph{Prefijos del sistema internacional}.
        \url{https://es.wikipedia.org/wiki/Prefijos_del_Sistema_Internacional}

\end{itemize}

\textbf{Lectura obligatoria:}

\vspace{-2\topsep}
\begin{itemize}

    \itemsep2pt \parskip0pt \parsep0pt

    \item Apuntes de cátedra. Capitulo 1: Sistemas de Numeración, y Capitulo
        2: Unidades de Información. Disponible en \textit{PEDCO}:
        \url{https://pedco.uncoma.edu.ar/mod/url/view.php?id=203642}

\end{itemize}

\section{Sistema de numeración no posicional}

El sistema de numeración egipcio es \textbf{aditivo}, es decir, cada número se
calcula sumando el valor de los símbolos. A continuación se muestran los
símbolos y sus valores:

\begin{center}
    \begin{tabular}[t]{|c|c|c|c|c|c|c|}

        \hline
        El dios \emph{Heh} & Renacuajo & Dedo & Flor de loto & Cuerda
        enrollada & Grillete & Trazo\\

        \egmil{1} & \eghuntho{1} & \egtentho{1} & \egtho{1} & \eghun{1} &
        \egten{1} & \egone{1}\\

        \hline
        1\,000\,000 & 100\,000 & 10\,000 & 1\,000 & 100 & 10 & 1\\
        \hline

    \end{tabular}
\end{center}

Por ejemplo, el número 13\,745 se podría escribir así:

\egyptify{0}{0}{1}{3}{7}{4}{5}

\begin{enumerate}

    \item Escribir los números que representen los siguientes símbolos
        egipcios:

    \begin{enumerate*}[itemjoin=\hspace{2em}]

        \item \egyptify{0}{0}{1}{0}{5}{0}{4}

        \item \egyptify{0}{0}{0}{1}{4}{1}{2}

    \end{enumerate*}

    \item Escribir en el sistema de numeración egipcio los siguientes números:

        \begin{enumerate*}[itemjoin=\hspace{2em}]

            \item 3\,421

            \item 1\,896

            \item 120\,218

        \end{enumerate*}

    \item La distancia promedio entre la tierra y el sol es de aproximadamente
        $149\,597\,870\,700$ metros\footnote{Esta distancia es conocida como
        \emph{unidad astronómica}.} ¿Puede expresar esta distancia utilizando
        el sistema de numeración Egipcio?

\end{enumerate}

\section{Sistema de numeración posicional}

\begin{enumerate}

    \item Descomponer los siguientes números en \emph{sumas de potencias de la
        base} y calcular el resultados de:

    \begin{enumerate*}[itemjoin=\hspace{2em}]

        \item $7\,249_{10}$

        \item $1\,0111_{2}$

        \item $127_{8}$

        \item $23\,9E_{16}$

        \item $10\,111_{3}$

        \item $10\,111_{9}$

    \end{enumerate*}

    \item Tras descomponer los números en sumas de potencias de la base ¿en
        qué base queda expresado el resultado?

\end{enumerate}

\subsection{Conversión entre sistemas de numeración posicional}

\begin{enumerate}

    \item Complete la tabla de conversiones \ref{tablaConversiones} de la
        página \pageref{tablaConversiones}. \label{ejTabla}

        Para convertir de decimal a otra base utilice el procedimiento de
        división; para convertir de otra base a decimal utilizar la
        descomposición en sumas de potencias de la base.

        \begin{enumerate}

            \item Una vez completada la tabla: ¿Encuentra algún patrón que
                permita una conversión rápida entre los sistemas binario,
                octal y hexadecimal?

        \end{enumerate}

        A continuación, para convertir de decimal a otra base utilizar el
        procedimiento de división; para convertir de otra base a decimal
        utilizar la descomposición en sumas de potencias de la base, y para
        convertir entre binario y octal/hexadecimal utilizar la tabla
        completada en el ejercicio \ref{ejTabla}.

    \item Convertir de decimal a binario, octal y hexadecimal:

    \begin{enumerate*}[itemjoin=\hspace{2em}]

        \item $132_{10}$

        \item $500_{10}$

        \item $27\,025_{10}$

    \end{enumerate*}

    \item Convertir de binario y hexadecimal a decimal:

    \begin{enumerate*}[itemjoin=\hspace{2em}]

        \item $00\,0011_{2}$

        \item $10\,1010_{2}$

        \item $10\,1111_{2}$

        \item $F4_{16}$

        \item $D\,3E_{16}$

        \item $EB\,AC_{16}$

    \end{enumerate*}

    \item Convertir de hexadecimal a binario:

    \begin{enumerate*}[itemjoin=\hspace{2em}]

        \item $FF_{16}$

        \item $B4_{16}$

        \item $1\,FC_{16}$

        \item $1\,A1_{16}$

        \item $23\,9E_{16}$

        \item $5F\,FF_{16}$

    \end{enumerate*}

    \item Convertir de binario a hexadecimal y octal:

    \begin{enumerate*}[itemjoin=\hspace{2em}]

        \item $1001\,0001\,1100\,1001_{2}$

        \item $0110\,1110\,1011\,1100_{2}$

    \end{enumerate*}

    \item En el siguiente número se desconoce un dígito representado con
        \emph{X}. ¿Qué valores puede tomar ese dígito desconocido?

    \begin{enumerate*}[itemjoin=\hspace{2em}]

        \item $621X43_{10}$

        \item $11X01_{2}$

        \item $43X21_{8}$

    \end{enumerate*}

    \item En el siguiente número se desconoce la base representada con
        \emph{Y}. ¿Cuál es el menor valor que puede tomar \emph{Y}?


    \begin{enumerate*}[itemjoin=\hspace{2em}]

        \item $6\,350_{Y}$

        \item $2\,031_{Y}$

        \item $348_{Y}$

    \end{enumerate*}

\end{enumerate}

\section{Unidades de información}

\begin{enumerate}

    \item Utilice la tabla \ref{tablaPrefijoSI} con prefijos del Sistema
        Internacional \emph{(SI)} de la página \pageref{tablaPrefijoSI} para
        expresar la distancia de 300 Megámetros \emph{(Mm)} en:

    \begin{enumerate*}[itemjoin=\hspace{2em}]

        \item Kilómetros~\emph{(km)}

        \item Metros~\emph{(m)}

        \item Milímetros~\emph{(mm)}

        \item Micrómetros~\emph{(µm)}

        \item Nanómetros~\emph{(nm)}

    \end{enumerate*}

    \item Exprese el tiempo de un año (considerando que un año tiene 365 días)
        en:

    \begin{enumerate*}[itemjoin=\hspace{2em}]

        \item Horas

        \item Minutos

        \item Segundos

        \item Milisegundos

        \item Microsegundos
            \hspace{2em} % Relleno para que el siguiente item bien acomodado

        \item Nanosegundos

    \end{enumerate*}

    \item Grafique la relación entre bytes y bits.

    \item Las siguientes cantidades son dadas en \textbf{prefijos
        binarios}(\url{http://es.wikipedia.org/wiki/Prefijo_binario}), exprese
        su cantidad equivalente en bytes y bits (Utilice la tabla
        \ref{tablaComparacionPrefijos} de la página
        \pageref{tablaComparacionPrefijos}).

    \begin{enumerate*}[itemjoin=\hspace{2em}]

        \item 64\emph{KiB}

        \item 15\emph{MiB}

        \item 4\emph{GiB}

        \item 2\emph{TiB}

        \item 9\emph{PiB}

        \item 3\emph{EiB}

    \end{enumerate*}

    \item Las siguientes cantidades son dadas en \textbf{prefijos
        decimales}(\url{https://es.wikipedia.org/wiki/Prefijos_del_Sistema_Internacional}),
        exprese su cantidad equivalente en bytes y bits  (Utilice la tabla
        \ref{tablaComparacionPrefijos} de la página
        \pageref{tablaComparacionPrefijos}).

    \begin{enumerate*}[itemjoin=\hspace{2em}]

        \item 64\emph{KB}

        \item 15\emph{MB}

        \item 4\emph{GB}

        \item 10\emph{TB}

        \item 9\emph{PB}

        \item 3\emph{EB}

    \end{enumerate*}

\end{enumerate}

\subsection{Unidades de información: resolver}

    \begin{enumerate}

        \item Al comprar un dispositivo o medio de almacenamiento secundario
            (disco rígido, pendrive, DVD) normalmente encontramos que el
            fabricante especifica la capacidad empleando prefijos decimales
            (\emph{KB}, \emph{MB}, \emph{TB}, etc.). Sin embargo,
            generalmente, un explorador de archivos muestra este dato
            utilizando prefijos binarios (\emph{KiB}, \emph{MiB}, \emph{TiB},
            etc.). Indique la capacidad que mostraría el explorador de
            archivos para dispositivos o medios de:

        \begin{enumerate*}[itemjoin=\hspace{2em}]

            \item 3\emph{MB}

            \item 4.7\emph{GB}

            \item 5\emph{TB}

            \item 8.5\emph{GB}

            \item 2\emph{PB}

        \end{enumerate*}

    \item Necesito comprar un pendrive para guardar 1990 fotos de 2 \emph{MiB}
        cada una.

        \begin{enumerate}

            \item ¿Cuántos \emph{GiB} de almacenamiento se necesitan?

            \item En un comercio hay pendrives disponibles de 2\emph{GB},
                4\emph{GB}, 8\emph{GB} y 16\emph{GB}, ¿cuál debería elegir de
                tal manera que pueda guardar todas las fotos y sobre el menor
                espacio posible?

        \end{enumerate}

    \item Aunque ambas nomenclaturas están estandarizadas, es normal que se
        utilice únicamente la de prefijos decimales, y debamos interpretar si
            se refiere a prefijo decimal o binario según el contexto.
            Supongamos que alguien envió un email diciendo: \emph{``He
            comprado un pendrive de 1GB y le he copiado una foto de 5MB"}.

        \begin{enumerate}

            \item ¿Cuántos bytes de capacidad tiene el pendrive?

            \item ¿Cuántos bytes tiene la foto?

        \end{enumerate}

    \end{enumerate}

\section{Integración}

    \begin{enumerate}

        \item Elabore un texto que compare los sistemas de numeración no
            posicional y posicional, tratados en este trabajo práctico.

            Para este ejercicio tenga en cuenta que:

            \begin{itemize}

                \item El texto descriptivo-comparativo tiene por objeto
                    comparar las características de dos o más seres,
                    destacando las semejanzas y las diferencias que hay entre
                    ellos.

                \item ¿Cómo se redacta un texto Descriptivo-Comparativo?
                    Cuando se comparan dos seres o dos objetos, sólo se deben
                    contrastar variables análogas, es decir, rasgos que
                    pertenecen a la misma clase. Podremos, por ejemplo,
                    comparar el tamaño (grande, pequeño), la forma (cuadrado,
                    rectangular), la materia (de vidrio, de metal).

                \item Un texto descriptivo-comparativo se puede esquematizar
                    mediante conectores que resalten los rasgos comunes y los
                    rasgos diferenciales, o bien mediante conectores que
                    contrasten los distintos rasgos de las realidades que se
                    comparan.

                \item Los textos que tienen estructura de
                    comparación-contraste utilizan conectores que manifiestan
                    semejanzas, es decir, paralelismo (igualmente, del mismo
                    modo, también, de la misma manera, asimismo...) o
                    diferencias, es decir, contraste (en cambio, sin embargo,
                    por el contrario, a diferencia de...).

            \end{itemize}

            \textbf{Fuente:} \emph{Lengua y Literatura, 2do de Bachillerato,
            2do Grado 1er Ciclo. Educación Media, Serie Ambar, Santillana.
            Pág. 149, MANUEL GARCÍA-CARTAGENA (Dominicano).}

\end{enumerate}

\begin{table}[h]

    \centering

    \caption{Tabla de conversiones}
    \label{tablaConversiones}

    \begin{tabular}{ | c | c | c | c | }
        \hline
        Decimal & Binario & Octal & Hexadecimal \\
        \hline
        0 & \hspace{8em} & \hspace{6em} & \hspace{6em} \\
        \hline
        1 & & & \\
        \hline
        2 & & & \\
        \hline
        3 & & & \\
        \hline
        4 & & & \\
        \hline
        5 & & & \\
        \hline
        6 & & & \\
        \hline
        7 & & & \\
        \hline
        8 & & & \\
        \hline
        9 & & & \\
        \hline
        10 & & & \\
        \hline
        11 & & & \\
        \hline
        12 & & & \\
        \hline
        13 & & & \\
        \hline
        14 & & & \\
        \hline
        15 & & & \\
        \hline
        16 & & & \\
        \hline
        234 & & & EA \\
        \hline
        \hspace{6em} & 1010\,1110 & & \\
        \hline
        & & 35 & \\
        \hline
        & 0010\,1011 & & \\
        \hline
        & & 70 & \\
        \hline
        & & & F0 \\
        \hline
        & 0001\,0100 & & \\
        \hline
        & 0010\,1000 & & \\
        \hline
        128 & & & \\
        \hline
        35 & & & \\
        \hline
        245 & & & \\
        \hline
        & & 42 & \\
        \hline
        & 010\, 100 & & \\
        \hline
        & & & 42 \\
        \hline
        & 0010\,0100 & & \\
        \hline
        255 & & & \\
        \hline
    \end{tabular}

\end{table}

\begin{table}[h]

    \centering

    \caption{Prefijos del Sistema Internacional}
    \label{tablaPrefijoSI}

    \begin{tabular}{ | c | c | c | }
        \hline
        Prefijo & Símbolo & Equivalencia a la unidad \\
        \hline
        \textbf{T} & tera & $10^{12}=1\,000^{4}$ \\
        \hline
        \textbf{G} & giga & $10^{9}=1\,000^{3}$ \\
        \hline
        \textbf{M} & mega & $10^{6}=1\,000^{2}$ \\
        \hline
        \textbf{K} & kilo & $10^{3}=1\,000^{1}$ \\
        \hline
        \multicolumn{2}{|c|}{\emph{sin prefijo}} & $10^{0}=1\,000^{0}=1$ \\
        \hline
        \textbf{m} & mili & $10^{-3}=1\,000^{-1}$ \\
        \hline
        \textbf{µ} & micro & $10^{-6}=1\,000^{-2}$ \\
        \hline
        \textbf{n} & nano & $10^{-9}=1\,000^{-3}$ \\
        \hline
    \end{tabular}

    \vspace{2\topsep}
    Ejemplo:
    \vspace{-1\topsep}
    \begin{itemize}

        \itemsep2pt \parskip0pt \parsep0pt
        \item   Un \emph{kilo}gramo son $10^{3}$ gramos.
        \item   Un \emph{nano}litro son $10^{-9}$ litros.

    \end{itemize}

\end{table}

\begin{table}[h]

    \centering

    \caption{Prefijos decimales y binarios}
    \label{tablaComparacionPrefijos}

    \begin{tabular}{ | c | c | }
        \hline
        \textbf{Prefijos decimales} &
        \textbf{prefijos binarios}\\
        \hline
        \hline
        $kilobyte(\emph{KB})=10^{3}\emph{bytes}=1\,000^{1}$\emph{bytes} &
        $kibibyte(\emph{KiB})=2^{10}\emph{bytes}=1\,024^{1}$\emph{bytes} \\
        \hline
        $megabyte(\emph{MB})=10^{6}\emph{bytes}=1\,000^{2}$\emph{bytes} &
        $mebibyte(\emph{MiB})=2^{20}\emph{bytes}=1\,024^{2}$\emph{bytes} \\
        \hline
        $gigabyte(\emph{GB})=10^{9}\emph{bytes}=1\,000^{3}$\emph{bytes} &
        $gibibyte(\emph{GiB})=2^{30}=1\,024^{3}$\emph{bytes} \\
        \hline
        $terabyte(\emph{TB})=10^{12\emph{bytes}}=1\,000^{4}$\emph{bytes} &
        $tebibyte(\emph{TiB})=2^{40}\emph{bytes}=1\,024^{4}$\emph{bytes} \\
        \hline
        $petabyte(\emph{PB})=10^{15}\emph{bytes}=1\,000^{5}$\emph{bytes} &
        $pebibyte(\emph{PiB})=2^{50}\emph{bytes}=1\,024^{5}$\emph{bytes} \\
        \hline
        $exabyte(\emph{EB})=10^{18}\emph{bytes}=1\,000^{6}$\emph{bytes} &
        $exbibyte(\emph{EiB})=2^{60}\emph{bytes}=1\,024^{6}$\emph{bytes} \\
        \hline
        $zettabyte(\emph{ZB})=10^{21}\emph{bytes}=1\,000^{7}$\emph{bytes} &
        $zebibyte(\emph{ZiB})=2^{70}\emph{bytes}=1\,024^{7}$\emph{bytes} \\
        \hline
        $yottabyte(\emph{YB})=10^{24}\emph{bytes}=1\,000^{8}$\emph{bytes} &
        $yobibyte(\emph{YiB})=2^{80}\emph{bytes}=1\,024^{8}$\emph{bytes} \\
        \hline
    \end{tabular}

    \vspace{2\topsep}
    Ejemplo:
    \vspace{-1\topsep}
    \begin{itemize}

        \itemsep2pt \parskip0pt \parsep0pt
        \item   Un \emph{kilo}byte son $1\,000^{1}$ bytes.
        \item   Un \emph{mebi}byte son $2^{20}$ bytes.

    \end{itemize}

\end{table}

\end{document}
