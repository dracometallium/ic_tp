%%% LaTeX Template: Article/Thesis/etc. with colored headings and special fonts
%%%
%%% Source: http://www.howtotex.com/
% vim: set spell spelllang=es syntax=tex :

\documentclass[12pt]{article}
\usepackage{styles/apuntes-estilo}
\usepackage{styles/egyptian}
\usepackage{fancyhdr,lastpage}
\usepackage{hyperref}
\usepackage[inline]{enumitem}

\def\maketitle{

\makeatletter
{\color{bl} \centering \huge \sc \textbf{ Trabajo práctico N° 1\\ \large
\vspace*{-8pt} \color{black} Sistemas de numeración y unidades de información \vspace*{8pt} }\par}
\makeatother

\makeatletter
% vim: set spell spelllang=es syntax=tex :
 {\centering \small 
    Introducción a la computación\\
    Departamento de Ingeniería de Computadoras \\
    Facultad de Informática - Universidad Nacional del Comahue \\
    \vspace{20pt} }
\makeatother

\vspace{-2.5cm}
\mbox{\hspace{-1cm}\includegraphics[width=3cm,height=3cm]{logos/uncoma.pdf}\hspace{13cm}
    \includegraphics[width=2.9cm,height=2.9cm]{logos/fai.pdf}}



}

% Custom headers and footers
\fancyhf{} % clear all header and footer fields
\fancypagestyle{plain}{\fancyhf{}}
\pagestyle{fancy}
\lhead{\footnotesize TP N° 1 - Sistemas de numeración y unidades de información}
\rhead{\footnotesize \thepage\ }	% "Page 1 of 2"

\def\ti#1#2{\texttt{#1} & #2 \\ }

\begin{document}

\thispagestyle{empty}
\maketitle
\setlength{\parindent}{1pt}

\textbf{Objetivo:} Comprender el sistema de numeración posicional, conversión
entre sistemas de distintas bases, y las unidades de información.

\textbf{Bibliografía básica:}

\vspace{-2\topsep}
\begin{itemize}

    \itemsep2pt \parskip0pt \parsep0pt

    \item   Andrew S. Tanenbaum. \emph{Organización de computadoras: un
        enfoque estructurado}. Tercera edición, México, Editorial Prentice
        Hall, 1992. ISBN 968-880-238-7. Disponible en biblioteca.

    \item   Wikipedia: \emph{Prefijo binario}.
        \url{http://es.wikipedia.org/wiki/Prefijo_binario}

    \item   Wikipedia: \emph{Prefijos del sistema internacional}.
        \url{https://es.wikipedia.org/wiki/Prefijos_del_Sistema_Internacional}

    \item   Apuntes de cátedra. Disponible en \it{PEDCO}.

\end{itemize}

\begin{enumerate}

    \item \textbf{Sistema de numeración no posicional}

        El sistema de numeración egipcio es \textbf{aditivo}, es decir, cada
        número se calcula sumando el valor de los símbolos. A continuación se
        muestran los símbolos y sus valores:

        \begin{center}
            \begin{tabular}[t]{|c|c|c|c|c|c|c|}
            \hline
            \egmil{1}&\eghuntho{1}&\egtentho{1}&\egtho{1}&\eghun{1}&\egten{1}&\egone{1}\\
            \hline
            1.000.000&100.000&10.000&1.000&100&10&1\\
            \hline
            \end{tabular}\\
        \end{center}

        Por ejemplo, el número 13.745 se podría escribir así:

        \egyptify{0}{0}{1}{3}{7}{4}{5}

    \begin{enumerate}

        \item Escribir los números que representen los siguientes símbolos
            egipcios:

        \begin{enumerate*}[itemjoin=\hspace{3em}]

            \item \egyptify{0}{0}{1}{0}{5}{0}{4}

            \item \egyptify{0}{0}{0}{1}{4}{1}{2}

        \end{enumerate*}

        \item Escribir en el sistema de numeración egipcio los siguientes
            números:

            \begin{enumerate*}[itemjoin=\hspace{3em}]

                \item 3421

                \item 1896

                \item 120218

            \end{enumerate*}

    \end{enumerate}

    \item \textbf{Sistema de numeración posicional}
    
        \begin{enumerate}

            \item Descomponer los siguientes números en \emph{sumas de potencias
                de la base} y calcular el resultados de:
            
            \begin{enumerate*}[itemjoin=\hspace{3em}]

                \item $7249_{10}$
                
                \item $10111_{2}$
                
                \item $127_{8}$

                \item $239E_{16}$

                \item $10111_{3}$

                \item $10111_{9}$

            \end{enumerate*}

            \item Tras descomponer los números en sumas de potencias de la
                base, ¿en qué base queda expresado el resultado?

        \end{enumerate}

    \item \textbf{Conversión entre sistemas de numeración posicional}

    \begin{enumerate}

        \item Complete la tabla de conversiones de la página 5.
            \label{ejTabla}
            
            Para convertir de decimal a otra base utilice el procedimiento de
            división; para convertir de otra base a decimal utilizar la
            descomposición en sumas de potencias de la base.
        
            \begin{enumerate}

                \item Una vez completada la tabla, describa con sus palabras
                    cuál es el patrón que observa para cada sistema binario,
                    octal y hexadecimal.
            
            \end{enumerate}

            A continuación, para convertir de decimal a otra base utilizar el
            procedimiento de división; para convertir de otra base a decimal
            utilizar la descomposición en sumas de potencias de la base, y
            para convertir entre binario y octal/hexadecimal utilizar las
            tabla elaborada en \ref{ejTabla}.

        \item Convertir de decimal a binario, octal y hexadecimal:

        \begin{enumerate*}[itemjoin=\hspace{3em}]
            
            \item $132_{10}$
                
            \item $500_{10}$

            \item $27025_{10}$

        \end{enumerate*}

        \item Convertir de binario y hexadecimal a decimal:

        \begin{enumerate*}[itemjoin=\hspace{3em}]
            
            \item $000011_{2}$

            \item $101010_{2}$

            \item $101111_{2}$

            \item $F4_{16}$

            \item $D3E_{16}$

            \item $EBAC_{16}$

        \end{enumerate*}

        \item Convertir de hexadecimal a binario:

        \begin{enumerate*}[itemjoin=\hspace{3em}]

            \item $FF_{16}$

            \item $B4_{16}$

            \item $1FC_{16}$

            \item $1A1_{16}$

            \item $239E_{16}$

            \item $5FFF_{16}$

        \end{enumerate*}

        \item Convertir de binario a hexadecimal y octal:

        \begin{enumerate*}[itemjoin=\hspace{3em}]

            \item $1001000111001001_{2}$

            \item $0110111010111100_{2}$

        \end{enumerate*}

    \end{enumerate}

\end{enumerate}

\end{document}
