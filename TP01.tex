%%% LaTeX Template: Article/Thesis/etc. with colored headings and special fonts
%%%
%%% Source: http://www.howtotex.com/
% vim: set spell spelllang=es syntax=tex :

\documentclass[12pt]{article}
\usepackage{styles/apuntes-estilo}
\usepackage{styles/egyptian}
\usepackage{fancyhdr,lastpage}
\usepackage{hyperref}
\usepackage[inline]{enumitem}

\def\maketitle{

\makeatletter
{\color{bl} \centering \huge \sc \textbf{ Trabajo práctico N° 1\\ \large
\vspace*{-8pt} \color{black} Sistemas de numeración y unidades de información \vspace*{8pt} }\par}
\makeatother

\makeatletter
% vim: set spell spelllang=es syntax=tex :
 {\centering \small 
    Introducción a la computación\\
    Departamento de Ingeniería de Computadoras \\
    Facultad de Informática - Universidad Nacional del Comahue \\
    \vspace{20pt} }
\makeatother

\vspace{-2.5cm}
\mbox{\hspace{-1cm}\includegraphics[width=3cm,height=3cm]{logos/uncoma.pdf}\hspace{13cm}
    \includegraphics[width=2.9cm,height=2.9cm]{logos/fai.pdf}}



}

% Custom headers and footers
\fancyhf{} % clear all header and footer fields
\fancypagestyle{plain}{\fancyhf{}}
\pagestyle{fancy}
\lhead{\footnotesize TP N° 1 - Sistemas de numeración y unidades de información}
\rhead{\footnotesize \thepage\ }	% "Page 1 of 2"

\def\ti#1#2{\texttt{#1} & #2 \\ }

\begin{document}

\thispagestyle{empty}
\maketitle
\setlength{\parindent}{1pt}

\textbf{Objetivo:} Comprender el sistema de numeración posicional, conversión
entre sistemas de distintas bases, y las unidades de información.

\textbf{Bibliografía básica:}

\vspace{-2\topsep}
\begin{itemize}

    \itemsep2pt \parskip0pt \parsep0pt

    \item Andrew S. Tanenbaum. \emph{Organización de computadoras: un enfoque
        estructurado}. Tercera edición, México, Editorial Prentice Hall, 1992.
        ISBN 968-880-238-7. Disponible en biblioteca.

    \item Wikipedia: \emph{Prefijo binario}.
        \url{http://es.wikipedia.org/wiki/Prefijo_binario}

    \item Wikipedia: \emph{Prefijos del sistema internacional}.
        \url{https://es.wikipedia.org/wiki/Prefijos_del_Sistema_Internacional}

    \item Apuntes de cátedra. Disponible en \it{PEDCO}.

\end{itemize}

\begin{enumerate}

    \item \textbf{Sistema de numeración no posicional}

        El sistema de numeración egipcio es \textbf{aditivo}, es decir, cada
        número se calcula sumando el valor de los símbolos. A continuación se
        muestran los símbolos y sus valores:

        \begin{center}
            \begin{tabular}[t]{|c|c|c|c|c|c|c|}
            \hline
            \egmil{1}&\eghuntho{1}&\egtentho{1}&\egtho{1}&\eghun{1}&\egten{1}&\egone{1}\\
            \hline
            1.000.000&100.000&10.000&1.000&100&10&1\\
            \hline
            \end{tabular}\\
        \end{center}

        Por ejemplo, el número 13.745 se podría escribir así:

        \egyptify{0}{0}{1}{3}{7}{4}{5}

    \begin{enumerate}

        \item Escribir los números que representen los siguientes símbolos
            egipcios:

        \begin{enumerate*}[itemjoin=\hspace{2em}]

            \item \egyptify{0}{0}{1}{0}{5}{0}{4}

            \item \egyptify{0}{0}{0}{1}{4}{1}{2}

        \end{enumerate*}

        \item Escribir en el sistema de numeración egipcio los siguientes
            números:

            \begin{enumerate*}[itemjoin=\hspace{2em}]

                \item 3421

                \item 1896

                \item 120218

            \end{enumerate*}

    \end{enumerate}

    \item \textbf{Sistema de numeración posicional}
    
        \begin{enumerate}

            \item Descomponer los siguientes números en \emph{sumas de potencias
                de la base} y calcular el resultados de:
            
            \begin{enumerate*}[itemjoin=\hspace{2em}]

                \item $7249_{10}$
                
                \item $10111_{2}$
                
                \item $127_{8}$

                \item $239E_{16}$

                \item $10111_{3}$

                \item $10111_{9}$

            \end{enumerate*}

            \item Tras descomponer los números en sumas de potencias de la
                base, ¿en qué base queda expresado el resultado?

        \end{enumerate}

    \item \textbf{Conversión entre sistemas de numeración posicional}

    \begin{enumerate}

        \item Complete la tabla de conversiones de la página 5.
            \label{ejTabla}
            
            Para convertir de decimal a otra base utilice el procedimiento de
            división; para convertir de otra base a decimal utilizar la
            descomposición en sumas de potencias de la base.
        
            \begin{enumerate}

                \item Una vez completada la tabla, describa con sus palabras
                    cuál es el patrón que observa para cada sistema binario,
                    octal y hexadecimal.
            
            \end{enumerate}

            A continuación, para convertir de decimal a otra base utilizar el
            procedimiento de división; para convertir de otra base a decimal
            utilizar la descomposición en sumas de potencias de la base, y
            para convertir entre binario y octal/hexadecimal utilizar las
            tabla elaborada en \ref{ejTabla}.

        \item Convertir de decimal a binario, octal y hexadecimal:

        \begin{enumerate*}[itemjoin=\hspace{2em}]
            
            \item $132_{10}$
                
            \item $500_{10}$

            \item $27025_{10}$

        \end{enumerate*}

        \item Convertir de binario y hexadecimal a decimal:

        \begin{enumerate*}[itemjoin=\hspace{2em}]
            
            \item $000011_{2}$

            \item $101010_{2}$

            \item $101111_{2}$

            \item $F4_{16}$

            \item $D3E_{16}$

            \item $EBAC_{16}$

        \end{enumerate*}

        \item Convertir de hexadecimal a binario:

        \begin{enumerate*}[itemjoin=\hspace{2em}]

            \item $FF_{16}$

            \item $B4_{16}$

            \item $1FC_{16}$

            \item $1A1_{16}$

            \item $239E_{16}$

            \item $5FFF_{16}$

        \end{enumerate*}

        \item Convertir de binario a hexadecimal y octal:

        \begin{enumerate*}[itemjoin=\hspace{2em}]

            \item $1001000111001001_{2}$

            \item $0110111010111100_{2}$

        \end{enumerate*}

    \end{enumerate}

    \item \textbf{Averiguar}
        
    \begin{enumerate}

        \item En el siguiente número se desconoce un dígito representado con
            \emph{X}. ¿Qué valores puede tomar ese dígito desconocido?

        \begin{enumerate*}[itemjoin=\hspace{2em}]

            \item $621X43_{10}$

            \item $11X01_{2}$

            \item $43X21_{8}$

        \end{enumerate*}

        \item En el siguiente número se desconoce la base representada con
            \emph{Y}. ¿Cuál es el menor valor que puede tomar \emph{Y}?


        \begin{enumerate*}[itemjoin=\hspace{2em}]

            \item $6350_{Y}$

            \item $2031_{Y}$

            \item $348_{Y}$

        \end{enumerate*}

    \end{enumerate}

    \item \textbf{Unidades de información}

    \begin{enumerate}

        \item Utilice la tabla con prefijos del Sistema Internacional
            \emph{(SI)} de la página 6 para expresar la distancia de 300
            Megámetros \emph{(Mm)} en:

        \begin{enumerate*}[itemjoin=\hspace{2em}]

            \item Kilómetros \emph{(km)}

            \item Metros \emph{(m)}

            \item Milímetros \emph{(mm)}

            \item Micrómetros \emph{(µm)}

            \item Nanómetros \emph{(nm)}

        \end{enumerate*}

        \item Exprese el tiempo de un año (considerando que un año tiene 365
            días) en:

        \begin{enumerate*}[itemjoin=\hspace{2em}]

            \item Horas

            \item Minutos

            \item Segundos 

            \item Milisegundos

            \item Microsegundos

            \item Nanosegundos

        \end{enumerate*}

        \item Grafique la relación entre bytes y bits.

        \item Las siguientes cantidades son dadas en \textbf{prefijos
            binarios}(\url{http://es.wikipedia.org/wiki/Prefijo_binario}),
            exprese su cantidad equivalente en bytes y bits. (Utilice la tabla
            de la página 6).

        \begin{enumerate*}[itemjoin=\hspace{2em}]

            \item 64\emph{KiB}

            \item 15\emph{MiB}

            \item 4\emph{GiB}

            \item 2\emph{TiB}

            \item 9\emph{PiB}

            \item 3\emph{EiB}

        \end{enumerate*}

    \item Las siguientes cantidades son dadas en \textbf{prefijos
        decimales}(\url{https://es.wikipedia.org/wiki/Prefijos_del_Sistema_Internacional}),
        exprese su cantidad equivalente en bytes y bits. (Utilice la tabla de
        la página 6).

        \begin{enumerate*}[itemjoin=\hspace{2em}]

            \item 64\emph{KB}

            \item 15\emph{MB}

            \item 4\emph{GB}

            \item 10\emph{TB}

            \item 9\emph{PB}

            \item 3\emph{EB}

        \end{enumerate*}

    \end{enumerate}

    \item \textbf{Unidades de información: resolver}

        \begin{enumerate}

            \item Al comprar un dispositivo o medio de almacenamiento
                secundario (disco rígido, pendrive, DVD) normalmente
                encontramos que el fabricante especifica la capacidad
                empleando prefijos decimales (\emph{KB}, \emph{MB}, \emph{TB},
                etc.). Sin embargo, generalmente, un explorador de archivos
                muestra este dato utilizando prefijos binarios (\emph{KiB},
                \emph{MiB}, \emph{TiB}, etc.). Indique la capacidad que
                mostraría el explorador de archivos para dispositivos o medios
                de:

            \begin{enumerate*}[itemjoin=\hspace{2em}]

                \item 3\emph{MB}

                \item 4.7\emph{GB}

                \item 5\emph{TB}

                \item 8.5\emph{GB}

                \item 2\emph{PB}

            \end{enumerate*}

        \item Necesito comprar un pendrive para guardar 1990 fotos de 2 \emph{MiB}
                cada una.

            \begin{enumerate}

                \item ¿Cuántos \emph{GiB} de almacenamiento se necesitan? 

                \item En un comercio hay pendrives disponibles de 2\emph{GB},
                    4\emph{GB}, 8\emph{GB} y 16\emph{GB}, ¿cuál debería elegir de
                    tal manera que pueda guardar todas las fotos y sobre el menor
                    espacio posible? 

            \end{enumerate}

        \item Aunque ambas nomenclaturas están estandarizadas, es normal que se
            utilice únicamente la de prefijos decimales, y debamos interpretar si
                se refiere a prefijo decimal o binario según el contexto.
                Supongamos que alguien envió un email diciendo: \emph{``He comprado
                un pendrive de 1GB y le he copiado una foto de 5MB"}.

            \begin{enumerate}

                \item ¿Cuántos bytes de capacidad tiene el pendrive? 
                    
                \item ¿Cuántos bytes tiene la foto? 

            \end{enumerate}

        \end{enumerate}

    \item \textbf{Integración}

        \begin{enumerate}

            \item Elabore un texto que compare los sistemas de numeración no
                posicional y posicional, tratados en este trabajo
                práctico.
                
                Para este ejercicio tenga en cuenta que:

                \begin{itemize}

                    \item El texto descriptivo-comparativo tiene por objeto
                        comparar las características de dos o más seres,
                        destacando las semejanzas y las diferencias que hay
                        entre ellos.

                    \item ¿Cómo se redacta un texto Descriptivo-Comparativo?
                        Cuando se comparan dos seres o dos objetos, sólo se
                        deben contrastar variables análogas, es decir, rasgos
                        que pertenecen a la misma clase. Podremos, por
                        ejemplo, comparar el tamaño (grande, pequeño), la
                        forma (cuadrado, rectangular), la materia (de vidrio,
                        de metal).

                    \item Un texto descriptivo-comparativo se puede
                        esquematizar mediante conectores que resalten los
                        rasgos comunes y los rasgos diferenciales, o bien
                        mediante conectores que contrasten los distintos
                        rasgos de las realidades que se comparan.

                    \item Los textos que tienen estructura de
                        comparación-contraste utilizan conectores que
                        manifiestan semejanzas, es decir, paralelismo
                        (igualmente, del mismo modo, también, de la misma
                        manera, asimismo...) o diferencias, es decir,
                        contraste (en cambio, sin embargo, por el contrario, a
                        diferencia de...).

                \end{itemize}

                \textbf{Fuente:} \emph{Lengua y Literatura, 2do de
                Bachillerato, 2do Grado 1er Ciclo. Educación Media, Serie
                Ambar, Santillana. Pág. 149, MANUEL GARCÍA-CARTAGENA
                (Dominicano).}

    \end{enumerate}

\end{enumerate}

\end{document}
