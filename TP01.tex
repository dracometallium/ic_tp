%%% LaTeX Template: Article/Thesis/etc. with colored headings and special fonts
%%%
%%% Source: http://www.howtotex.com/
% vim: set spell spelllang=es syntax=tex :

\documentclass[12pt]{article}
\usepackage{styles/apuntes-estilo}
\usepackage{styles/egyptian}
\usepackage{fancyhdr,lastpage}
\usepackage{hyperref}

\def\maketitle{

\makeatletter
{\color{bl} \centering \huge \sc \textbf{ Trabajo práctico N° 1\\ \large
\vspace*{-8pt} \color{black} Sistemas de numeración y unidades de información \vspace*{8pt} }\par}
\makeatother

\makeatletter
% vim: set spell spelllang=es syntax=tex :
 {\centering \small 
    Introducción a la computación\\
    Departamento de Ingeniería de Computadoras \\
    Facultad de Informática - Universidad Nacional del Comahue \\
    \vspace{20pt} }
\makeatother

\vspace{-2.5cm}
\mbox{\hspace{-1cm}\includegraphics[width=3cm,height=3cm]{logos/uncoma.pdf}\hspace{13cm}
    \includegraphics[width=2.9cm,height=2.9cm]{logos/fai.pdf}}



}

% Custom headers and footers
\fancyhf{} % clear all header and footer fields
\fancypagestyle{plain}{\fancyhf{}}
\pagestyle{fancy}
\lhead{\footnotesize TP N° 1 - Sistemas de numeración y unidades de información}
\rhead{\footnotesize \thepage\ }	% "Page 1 of 2"

\def\ti#1#2{\texttt{#1} & #2 \\ }

\begin{document}

\thispagestyle{empty}
\maketitle
\setlength{\parindent}{1pt}

\textbf{Objetivo:} Comprender el sistema de numeración posicional, conversión
entre sistemas de distintas bases, y las unidades de información.

\textbf{Bibliografía básica:}

\vspace{-2\topsep}
\begin{itemize}

    \itemsep2pt \parskip0pt \parsep0pt

    \item   Andrew S. Tanenbaum. \emph{Organización de computadoras: un
        enfoque estructurado}. Tercera edición, México, Editorial Prentice
        Hall, 1992. ISBN 968-880-238-7. Disponible en biblioteca.

    \item   Wikipedia: \emph{Prefijo binario}.
        \url{http://es.wikipedia.org/wiki/Prefijo_binario}

    \item   Wikipedia: \emph{Prefijos del sistema internacional}.
        \url{https://es.wikipedia.org/wiki/Prefijos_del_Sistema_Internacional}

\item   Apuntes de cátedra. Disponible en \it{PEDCO}.

\end{itemize}

\egmil{1}
\eghuntho{1}
\egtentho{1}
\egtho{1}
\eghun{1}
\egten{1}
\egone{1}


\begin{enumerate}

    \item En un sistema GNU/Linux, cree un archivo de texto pequeño (al menos
        dos líneas) de contenido arbitrario con su editor favorito, guárdelo
        como {\tt archivo1}.

    \item ¿Qué formato interno indica el comando {\tt file} sobre
        \emph{archivo1}? 

    \item ¿ Qué aplicaciones podría utilizar para visualizar el contenido de
        \emph{archivo1}?

    \item Observe el contenido de \emph{archivo1} utilizando: ``{\tt od -c
        archivo1}''. ¿Observa información adicional a la que vemos a través de
        otras aplicaciones, como puede ser \emph{cat} o algún editor de texto?
        ¿Qué representa esa información?

    \item Observe el contenido de \emph{archivo1} utilizando: ``{\tt od -b
        archivo1}''. En este caso observamos el contenido numérico de cada
        byte. Cada grupo de tres dígitos octales mostrado, indica el contenido
        de un byte. De este modo, podemos observar el flujo de bits que
        componen un archivo de texto.

\end{enumerate}

Ahora cabe preguntarse. ¿Cómo es que un byte almacenado con valor octal 141
(01100001), al ser visualizado por un editor se ve como el caracter ``a''?
Esto sucede porque la aplicación que lo muestra hace una interpretación de
esos bits utilizando un código. Observe la salida de ``{\tt file archivo1}''
nuevamente y la referencia al sistema de codificación utilizado por el
archivo. 

\begin{enumerate}

    \item Observando el contenido numérico de cada byte nuevamente ({\tt od -b
    archivo1}), compare dichos valores con la tabla de caracteres ASCII
    (http://en.wikipedia.org/wiki/ASCII) y corrobore el mapeo con la salida de
    {\tt od -c archivo1}  para el mismo archivo.

    \item ¿Cómo se llaman y qué función cumplen el/los caracter/es de control
    observados en la salidas anteriores? 

\end{enumerate}

Hasta aquí no hemos enfrentado ningún problema de interoperabilidad con otros
sistemas operativos. En principio el formato interno de archivos de texto no
pareciera representar ningún desafío a la hora de ser interpretados en
aplicaciones de uno u otro sistema operativo, sin embargo el uso de los
caracteres de control que representan el fin de línea es el comienzo de la
discordia. 

En los sistemas GNU/Linux el caracter de control que indica el fin de una
línea, en un archivo de texto es LF (Line Feed), visualizado como
``\textbackslash n''
\footnote{http://en.wikipedia.org/wiki/ASCII\#ASCII\_control\_code\_chart}.
Mientras que en los sistemas Windows (C), será la combinación de CR (Carriage
Return) \textbackslash r, y LF (Line Feed) lo que determine el fin de una
línea.

Vemos más de cerca esta ``insignificante'' discrepancia, y analicemos las
implicaciones para los usuarios normales. 

\begin{enumerate}

    \item Abra el \emph{archivo1} con el editor \emph{Vim} y, en modo normal,
    ejecute: {\tt :set ff?}. Este comando nos mostrará qué tipo de {\it fin de
    línea} se está utilizando para leer el \emph{archivo1}.  Las opciones
    pueden ser \emph{unix}, dos o mac (algunas versiones del sistema operativo
    Mac OS utilizan sólo CR (\emph{Carriage Return}) para finalizar una
    línea). ``ff'' se corresponde con ``fileformat''.

    \begin{itemize}

        \item Vuelva a observar con {\tt od -c} el caracter de control de fin
        de línea de \emph{archivo1} y verifique lo mostrado por \emph{Vim}. 

        \item Vuelva a observar qué formato de archivo indica el comando {\tt
        file} sobre \emph{archivo1}. 

    \end{itemize} 

    \item Abra el \emph{archivo1} con el editor Vim y en modo normal, ejecute:
    {\tt :bufdo! set
    ff=dos|w}\footnote{http://Vim.wikia.com/wiki/File\_format}. 

    \begin{itemize}

        \item Consulte nuevamente el formato en uso: {\tt :set ff?}. ¿Cambió?

        \item Vuelva a observar con {\tt od -c} el caracter de control de fin
        de línea de \emph{archivo1} y verifique lo mostrado por \emph{Vim}. 

        \item Vuelva a observar qué formato de archivo indica el comando {\tt
        file} sobre \emph{archivo1}. ¿Cambió algo con respecto a la versión original? 

        \item ¿Qué función cumple entonces el comando {\tt :bufdo!} utilizado
        al comienzo de este ejercicio?

    \end{itemize}

\end{enumerate}

Teniendo en cuenta todo lo visto hasta el momento, cabe ahora preguntarnos: 

\begin{enumerate}

    \item ¿Cómo afecta esta discrepancia a los usuarios normales? Imagine un
    usuario de un sistema GNU/Linux que manda un correo electrónico con un
    archivo de texto adjunto a un usuario de un sistema \emph{Microsoft
    Windows} (C). ¿Qué sucederá cuando este último quiera abrir el archivo de
    texto, digamos con \emph{Notepad}?

    \item Observe editores gráficos como \emph{gedit}. ¿Qué opciones relativas
    a estas cuestiones de interoperabilidad poseen a la hora de guardar un
    archivo? ¿Cuál sería el equivalente en  \emph{gedit} a la sentencia {\tt
    :bufdo! set ff=dos|w } de \emph{Vim}? 

\end{enumerate}
     
Este es el ejemplo más sencillo que podemos analizar acerca de cómo distintas
aplicaciones y distintos sistemas operativos pueden hacer interpretaciones
diferentes del formato interno de un archivo. Imagine ahora que en lugar de un
archivo de texto, hablamos de un archivo con formato enriquecido como puede
ser el generado por \emph{Microsoft Office}, o \emph{LibreOffice}. ¿Cuántas
discrepancias podrán surgir en la interpretación del formato? Parte de los
objetivos de esta asignatura será vislumbrar esta clase de inconvenientes,
cómo afecta esto a los usuarios y qué podemos hacer los administradores para
ayudarlos. 

\end{document}
