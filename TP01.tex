%%% LaTeX Template: Article/Thesis/etc. with colored headings and special fonts
%%%
%%% Source: http://www.howtotex.com/
% vim: set spell spelllang=es syntax=tex :

\documentclass[12pt]{article}
\usepackage{styles/apuntes-estilo}
\usepackage{styles/egyptian}
\usepackage{fancyhdr,lastpage}
\usepackage{hyperref}
\usepackage[inline]{enumitem}
\usepackage{xurl}
\usepackage[T1]{fontenc} 

\def\maketitle{

\makeatletter{
    \color{blue} \centering \huge \sc
    \textbf{
        Trabajo práctico N° 1\\
        \large \vspace*{-8pt} \color{black}
        Sistemas de numeración
        \vspace*{8pt}
    }\\
    \small Fecha de finalización: 29 de marzo
    \par
}

\makeatother

\makeatletter
% vim: set spell spelllang=es syntax=tex :
 {\centering \small 
    Introducción a la computación\\
    Departamento de Ingeniería de Computadoras \\
    Facultad de Informática - Universidad Nacional del Comahue \\
    \vspace{20pt} }
\makeatother

\vspace{-2.5cm}
\mbox{\hspace{-1cm}\includegraphics[width=3cm,height=3cm]{logos/uncoma.pdf}\hspace{13cm}
    \includegraphics[width=2.9cm,height=2.9cm]{logos/fai.pdf}}



}

% Custom headers and footers
\fancyhf{} % clear all header and footer fields
\fancypagestyle{plain}{\fancyhf{}}
\pagestyle{fancy}
\lhead{\footnotesize TP N° 1 - Sistemas de numeración}
\rhead{\footnotesize \thepage\ }

\def\ti#1#2{\texttt{#1} & #2 \\ }

\begin{document}

\thispagestyle{empty}
\maketitle
\setlength{\parindent}{1pt}

\textbf{Objetivo:} Comprender el sistema de numeración posicional, y conversión
entre sistemas de distintas bases.

\textbf{Lectura obligatoria:}

\vspace{-2\topsep}
\begin{itemize}

    \itemsep2pt \parskip0pt \parsep0pt

    \item Apuntes de cátedra. Capitulo 1: Sistemas de Numeración. Disponible
        en: \url{https://egrosclaude.github.io/IC/IC-notes.pdf}

\end{itemize}

\section{Sistema de numeración no posicional}

El sistema de numeración egipcio es \textbf{aditivo}, es decir, cada número se
calcula sumando el valor de los símbolos. A continuación se muestran los
símbolos y sus valores:

\begin{center}
    \begin{tabular}[t]{|c|c|c|c|c|c|c|}

        \hline
        El dios \emph{Heh} & Renacuajo & Dedo & Flor de loto & Cuerda
        enrollada & Grillete & Trazo\\

        \egmil{1} & \eghuntho{1} & \egtentho{1} & \egtho{1} & \eghun{1} &
        \egten{1} & \egone{1}\\

        \hline
        1\,000\,000 & 100\,000 & 10\,000 & 1\,000 & 100 & 10 & 1\\
        \hline

    \end{tabular}
\end{center}

Por ejemplo, el número 13\,745 se podría escribir así:

\egyptify{0}{0}{1}{3}{7}{4}{5}

\begin{enumerate}[resume]

    \item Escribir los números que representen los siguientes símbolos
        egipcios:

    \begin{enumerate*}[itemjoin=\hspace{2em}]

        \item \egyptify{0}{0}{1}{0}{5}{0}{4}

        \item \egyptify{0}{0}{0}{1}{4}{1}{2}

    \end{enumerate*}

    \item Escribir en el sistema de numeración egipcio los siguientes números:

        \begin{enumerate*}[itemjoin=\hspace{2em}]

            \item 3\,421

            \item 1\,896

        \end{enumerate*}

    \item La distancia promedio entre la tierra y el sol es de aproximadamente
        $149\,597\,870\,700$ metros\footnote{Esta distancia es conocida como
        \emph{unidad astronómica}.} ¿Puede expresar esta distancia utilizando
        el sistema de numeración Egipcio? ¿Qué problemas pueden surgir?

\end{enumerate}

\section{Sistema de numeración posicional}

\begin{enumerate}[resume]

    \item Descomponer los siguientes números en \emph{sumas de potencias de la
        base} y calcular el resultados de:

    \begin{enumerate*}[itemjoin=\hspace{2em}]

        \item $7\,249_{10}$

        \item $1\,0111_{2}$

        \item $125_{6}$

        \item $23\,9E_{16}$

    \end{enumerate*}

    \item Tras descomponer los números en sumas de potencias de la base ¿en
        qué base queda expresado el resultado?

\end{enumerate}

\subsection{Conversión entre sistemas de numeración posicional}

\begin{enumerate}[resume]

    \item Realice las conversiones entre los sistemas de numeración posicional          
          decimal, binario, octal y hexadecimal que a continuación se describen.
    
        \begin{enumerate}
        
        		\item Complete toda la tabla de conversiones, que es la 
              	  tabla \ref{tablaConversiones} de la 
              	  página \pageref{tablaConversiones}. \label{ejTabla} 
              	  Para ello, tenga en cuenta que:
          \begin{itemize}
             \item para convertir de base decimal a otra base: utilice el
        			   procedimiento de división;
           	\item para convertir de una base a base decimal
        			  utilice la descomposición en sumas de potencias de la base; y 
        		\item para convertir entre binario y octal/hexadecimal aproveche las
        			  conversiones que ya ha completado en la misma 
        			  tabla \ref{tablaConversiones}.
          \end{itemize}

            \item Una vez completada la tabla: ¿Encuentra algún patrón que
                permita una conversión más rápida entre los siste~mas binario,
                octal y hexadecimal?

        \end{enumerate}

    \item Convertir de hexadecimal a binario:

    \begin{enumerate*}[itemjoin=\hspace{2em}]

        \item $FF_{16}$

        \item $B4_{16}$

        \item $23\,9E_{16}$

        \item $5F\,FF_{16}$

    \end{enumerate*}

    \item Convertir de binario a hexadecimal y octal:

    \begin{enumerate*}[itemjoin=\hspace{2em}]

        \item $1001\,0001\,1100\,1001_{2}$

        \item $0110\,1110\,1011\,1100_{2}$

    \end{enumerate*}

    \item En los siguientes números se desconoce un dígito (representado con
        \emph{X}) ¿Qué valores puede tomar ese dígito desconocido en cada caso?

    \begin{enumerate*}[itemjoin=\hspace{2em}]

        \item $621X43_{10}$

        \item $11X01_{2}$

        \item $43X21_{9}$

    \end{enumerate*}

    \item En los siguientes números se desconoce la base (representada con
        \emph{Y}) ¿Cuál es el menor valor que puede tomar \emph{Y} en cada
        caso?

    \begin{enumerate*}[itemjoin=\hspace{2em}]

        \item $6\,350_{Y}$

        \item $2\,031_{Y}$

        \item $348_{Y}$

    \end{enumerate*}

\end{enumerate}

\begin{table}[h]

    \centering

    \caption{Tabla de conversiones}
    \label{tablaConversiones}

    \begin{tabular}{ | c | c | c | c | }
        \hline
        Decimal & Binario & Octal & Hexadecimal \\
        \hline
        0 & \hspace{8em} & \hspace{6em} & \hspace{6em} \\
        \hline
        1 & & & \\
        \hline
        2 & & & \\
        \hline
        3 & & & \\
        \hline
        4 & & & \\
        \hline
        5 & & & \\
        \hline
        6 & & & \\
        \hline
        7 & & & \\
        \hline
        8 & & & \\
        \hline
        9 & & & \\
        \hline
        10 & & & \\
        \hline
        11 & & & \\
        \hline
        12 & & & \\
        \hline
        13 & & & \\
        \hline
        14 & & & \\
        \hline
        15 & & & \\
        \hline
        16 & & & \\
        \hline
        234 & & & EA \\
        \hline
        \hspace{6em} & 1010\,1110 & & \\
        \hline
        & & 35 & \\
        \hline
        & 0010\,1011 & & \\
        \hline
        & & 70 & \\
        \hline
        & & & F0 \\
        \hline
        & 0001\,0100 & & \\
        \hline
        & 0010\,1000 & & \\
        \hline
        128 & & & \\
        \hline
        35 & & & \\
        \hline
        245 & & & \\
        \hline
        & & 42 & \\
        \hline
        & 010\, 100 & & \\
        \hline
        & & & 42 \\
        \hline
        & 0010\,0100 & & \\
        \hline
        255 & & & \\
        \hline
    \end{tabular}

\end{table}

\end{document}
