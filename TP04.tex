%%% LaTeX Template: Article/Thesis/etc. with colored headings and special fonts
%%%
%%% Source: http://www.howtotex.com/
% vim: set spell spelllang=es syntax=tex :

\documentclass[12pt]{article}
\usepackage{styles/apuntes-estilo}
\usepackage{fancyhdr,lastpage}
\usepackage{hyperref}
\usepackage[inline]{enumitem}
\usepackage{xurl}

\def\maketitle{

\makeatletter{
    \color{bl} \centering \huge \sc
    \textbf{
        Trabajo práctico N° 4\\
        \large \vspace*{-8pt} \color{black}
        Representación de la información
        \vspace*{8pt}
    }\\
    \small Fecha de finalización: 22 de abril
    \par
}

\makeatother

\makeatletter
% vim: set spell spelllang=es syntax=tex :
 {\centering \small 
    Introducción a la computación\\
    Departamento de Ingeniería de Computadoras \\
    Facultad de Informática - Universidad Nacional del Comahue \\
    \vspace{20pt} }
\makeatother

\vspace{-2.5cm}
\mbox{\hspace{-1cm}\includegraphics[width=3cm,height=3cm]{logos/uncoma.pdf}\hspace{13cm}
    \includegraphics[width=2.9cm,height=2.9cm]{logos/fai.pdf}}



}

% Custom headers and footers
\fancyhf{} % clear all header and footer fields
\fancypagestyle{plain}{\fancyhf{}}
\pagestyle{fancy}
\lhead{\footnotesize TP N° 4 - Representación de la información}
\rhead{\footnotesize \thepage\ }

\def\ti#1#2{\texttt{#1} & #2 \\ }

\begin{document}

\thispagestyle{empty}
\maketitle
\setlength{\parindent}{1pt}

\textbf{Objetivo:} comprender la representación binaria de números enteros
representados en \emph{Complemento a dos}.

\textbf{Recursos web:}

\vspace{-2\topsep}
\begin{itemize}

    \item Wikipedia: \emph{Complemento a 2}:
        \url{https://en.wikipedia.org/wiki/Two\%27s_complement}

\end{itemize}

\textbf{Lectura obligatoria:}

\vspace{-2\topsep}
\begin{itemize}

    \itemsep2pt \parskip0pt \parsep0pt

    \item Apuntes de cátedra. Capitulo 3: Representación de la Información.
        Disponible en: \url{https://egrosclaude.github.io/IC/IC-notes.pdf}

\end{itemize}

\section{Operaciones aritméticas de números enteros}

\begin{enumerate}

    \item Dados los siguientes números representados en Complemento a 2 con 6
        bits, efectuar las siguientes restas utilizando el mecanismo donde la
        resta se transforma en una suma: $A-B = A+(-B)$.

        \begin{enumerate*}[itemjoin=\hspace{2em}]

            \item $00\,1010 - 00\,0110$

            \item $01\,0000 - 00\,0001$

            \item $01\,1100 - 11\,1111$

        \end{enumerate*}

    \item Determinar cuáles de las siguientes operaciones producen overflow,
        considerando una representación en \emph{complemento a 2} con
        \textbf{8 bits}:

        \begin{enumerate*}[itemjoin=\hspace{2em}]

            \item $0100\,1111 + 0011\,1100$

            \item $0101\,1111 + 1011\,1100$

            \item $1010\,0100 + 1101\,1000$

        \end{enumerate*}

    \item Elija un numero $N$ entre 33 y 50 y complete la siguiente tabla,
        realizando la \textbf{división entera} del número decimal y luego
        representándolo en binario:

        \begin{center}

            \begin{tabular}[t]{|c|c|c|}

            \cline{2-3}

            \multicolumn{1}{c|}{}& \textbf{Decimal} & \textbf{Binario} \\

            \hline

                $N$ & \hspace{9em}~ &\hspace{9em}~\\

            \hline

                $N/(2^1)$ & ~ &~\\

            \hline

                $N/(2^2)$ & ~ &~\\

            \hline

                $N/(2^3)$ & ~ &~\\

            \hline

                $N/(2^4)$ & ~ &~\\

            \hline

                $N/(2^5)$ & ~ &~\\

            \hline

            \end{tabular}

        \end{center}

        \begin{enumerate}

            \item ¿De qué manera sencilla se puede multiplicar y dividir por diez un
                número representado en base 10 sin realizar cálculo alguno?

            \item ¿Puede deducir algún mecanismo sencillo para dividir por dos un
                número representado en binario?

            \item ¿Puede deducir algún mecanismo sencillo para multiplicar por dos un
                número representado en binario?

        \end{enumerate}

\end{enumerate}
\end{document}
