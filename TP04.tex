%%% LaTeX Template: Article/Thesis/etc. with colored headings and special fonts
%%%
%%% Source: http://www.howtotex.com/
% vim: set spell spelllang=es syntax=tex :

\documentclass[12pt]{article}
\usepackage{styles/apuntes-estilo}
\usepackage{fancyhdr,lastpage}
\usepackage{hyperref}
\usepackage[inline]{enumitem}
\usepackage{xurl}
\usepackage{nameref}

\def\maketitle{

\makeatletter{
    \color{blue} \centering \huge \sc
    \textbf{
        Trabajo práctico N° 4\\
        \large \vspace*{-8pt} \color{black}
        Arquitectura y organización de computadoras
        \vspace*{8pt}
    }\\
    \small Fecha de finalización: 26 de mayo
    \par
}

\makeatother

\makeatletter
% vim: set spell spelllang=es syntax=tex :
 {\centering \small 
    Introducción a la computación\\
    Departamento de Ingeniería de Computadoras \\
    Facultad de Informática - Universidad Nacional del Comahue \\
    \vspace{20pt} }
\makeatother

\vspace{-2.5cm}
\mbox{\hspace{-1cm}\includegraphics[width=3cm,height=3cm]{logos/uncoma.pdf}\hspace{13cm}
    \includegraphics[width=2.9cm,height=2.9cm]{logos/fai.pdf}}



}

% Custom headers and footers
\fancyhf{} % clear all header and footer fields
\fancypagestyle{plain}{\fancyhf{}}
\pagestyle{fancy}
\lhead{\footnotesize TP N° 4 - Arquitectura y organización de computadoras}
\rhead{\footnotesize \thepage\ }

\def\ti#1#2{\texttt{#1} & #2 \\ }

\begin{document}

\thispagestyle{empty}
\maketitle
\setlength{\parindent}{1pt}

\textbf{Objetivo:} Comprender la organización y el funcionamiento básico de
una computadora simple. Se involucran conocimientos de los componentes
hardware y sus interacciones para ejecutar instrucciones.

\textbf{Recursos bibliográfico:}

\vspace{-2\topsep}
\begin{itemize}

    \itemsep2pt \parskip0pt \parsep0pt

    \item \emph{Andrew S. Tanenbaum}. Organización de computadoras: un enfoque
        estructurado. Cuarta edición, editorial Pearson Educación, 2000. ISBN
        970-170-399-5.

\end{itemize}

\textbf{Lectura obligatoria:}

\vspace{-2\topsep}
\begin{itemize}

    \itemsep2pt \parskip0pt \parsep0pt

    \item Apuntes de cátedra. Capitulo 5: Arquitectura y Organización de
        Computadoras. Disponible en \textit{PEDCO}:
        \url{https://pedco.uncoma.edu.ar/mod/url/view.php?id=203642}

\end{itemize}

\section{Modelo Computacional Binario Elemental (MCBE)}

\begin{enumerate}

    \item Con respecto a la memoria de la \emph{MCBE}, indique:

        \begin{enumerate}

            \item Cantidad de celdas de memoria.

            \item Tamaño de una celda de memoria en \emph{bits}
                y \emph{bytes}.

            \item Tamaño total en \emph{bytes}.

            \item Dirección de la primera y de última celda de memoria.

        \end{enumerate}

    \item Con respecto a la \emph{CPU} de la \emph{MCBE}, indique:

        \begin{enumerate}

            \item Registros y sus propósitos.

            \item ¿Qué representación y tamaño (en \emph{bits}) de números
                utilizan las instrucciones aritméticas?

            \item ¿En qué dirección de memoria debe ubicarse la primera
                instrucción del programa?
            
            \item ¿Qué efecto tendría ubicar los datos del programa a partir
                de la posición de memoria 0 y las instrucciones a
                continuación?

        \end{enumerate}

\end{enumerate}

\end{document}
