%%% LaTeX Template: Article/Thesis/etc. with colored headings and special fonts
%%%
%%% Source: http://www.howtotex.com/
% vim: set spell spelllang=es syntax=tex :

\documentclass[12pt]{article}
\usepackage{styles/apuntes-estilo}
\usepackage{fancyhdr,lastpage}
\usepackage{hyperref}
\usepackage[inline]{enumitem}
\usepackage{xurl}

\def\maketitle{

\makeatletter{
    \color{bl} \centering \huge \sc
    \textbf{
        Trabajo práctico N° 4\\
        \large \vspace*{-8pt} \color{black}
        Representación de la información
        \vspace*{8pt}
    }\\
    \small Fecha de finalización: 29 de Abril de 2022
    \par
}

\makeatother

\makeatletter
% vim: set spell spelllang=es syntax=tex :
 {\centering \small 
    Introducción a la computación\\
    Departamento de Ingeniería de Computadoras \\
    Facultad de Informática - Universidad Nacional del Comahue \\
    \vspace{20pt} }
\makeatother

\vspace{-2.5cm}
\mbox{\hspace{-1cm}\includegraphics[width=3cm,height=3cm]{logos/uncoma.pdf}\hspace{13cm}
    \includegraphics[width=2.9cm,height=2.9cm]{logos/fai.pdf}}



}

% Custom headers and footers
\fancyhf{} % clear all header and footer fields
\fancypagestyle{plain}{\fancyhf{}}
\pagestyle{fancy}
\lhead{\footnotesize TP N° 4 - Representación de la información}
\rhead{\footnotesize \thepage\ }

\def\ti#1#2{\texttt{#1} & #2 \\ }

\begin{document}

\thispagestyle{empty}
\maketitle
\setlength{\parindent}{1pt}

\textbf{Objetivo:} comprender la representación binaria de números de punto
(coma) fijo, y la suma de números enteros y punto fijo en binario.

\textbf{Recursos web:}

\vspace{-2\topsep}
\begin{itemize}

    \itemsep2pt \parskip0pt \parsep0pt
    \item Wikipedia: \emph{IEEE coma flotante}:
        \url{http://es.wikipedia.org/wiki/IEEE_coma_flotante}

    \item Wikipedia: \emph{Complemento a 2}:
        \url{https://en.wikipedia.org/wiki/Two\%27s_complement}

\end{itemize}

\textbf{Lectura obligatoria:}

\vspace{-2\topsep}
\begin{itemize}

    \itemsep2pt \parskip0pt \parsep0pt

    \item Apuntes de cátedra. Capitulo 3: Representación de la Información.
        Disponible en: \url{https://egrosclaude.github.io/IC/IC-notes.pdf}

\end{itemize}

\textbf{Nota}: La abreviatura ``Hex'' significa Hexadecimal, y el prefijo
``\textbf{0x}'' indica que un número está en hexadecimal.

\section{Operaciones aritméticas de números enteros}

\begin{enumerate}

    \item Dados los siguientes números representados en Complemento a 2 con 6
        bits, efectuar las siguientes restas utilizando el mecanismo donde la
        resta se transforma en una suma, es decir: $A-B = A+(-B)$.

        \begin{enumerate*}[itemjoin=\hspace{2em}]

            \item $00\,1010 - 00\,0110$

            \item $01\,0000 - 00\,0001$

            \item $01\,1100 - 11\,1111$

        \end{enumerate*}

    \item Determinar cuáles de las siguientes operaciones producen \textbf{overflow},
        considerando una representación en \emph{complemento a 2} con
        8 bits:

        \begin{enumerate*}[itemjoin=\hspace{2em}]

            \item $0100\,1111 + 0011\,1100$

            \item $0101\,1111 + 1011\,1100$

            \item $1010\,0100 + 1101\,1000$

        \end{enumerate*}

    \item Elija un numero $N$ entre 33 y 50 y complete la siguiente tabla,
        realizando la \textbf{división entera} del número decimal y luego
        representándolo en binario:

        \begin{center}

            \begin{tabular}[t]{|c|c|c|}

            \cline{2-3}

            \multicolumn{1}{c|}{}& \textbf{Decimal} & \textbf{Binario} \\

            \hline

                $N$ & \hspace{9em}~ &\hspace{9em}~\\

            \hline

                $N/(2^1)$ & ~ &~\\

            \hline

                $N/(2^2)$ & ~ &~\\

            \hline

                $N/(2^3)$ & ~ &~\\

            \hline

                $N/(2^4)$ & ~ &~\\

            \hline

                $N/(2^5)$ & ~ &~\\

            \hline

            \end{tabular}

        \end{center}

        \begin{enumerate}

            \item ¿De qué manera sencilla se puede multiplicar y dividir por diez un
                número representado en base 10 sin realizar cálculo alguno?

            \item ¿Puede deducir algún mecanismo sencillo para dividir por dos un
                numero representado en binario?

            \item ¿Puede deducir algún mecanismo sencillo para multiplicar por dos un
                numero representado en binario?

        \end{enumerate}

\end{enumerate}

\section{Representación de números reales}

\begin{enumerate}[resume]

    \item Representar los números reales en notación de \emph{punto Fijo} y
        \emph{complemento a 2}, utilizando 4 bits para la parte entera y 4
        para la parte fraccionaria:

        \begin{enumerate*}[itemjoin=\hspace{2em}]

            \item $1.75$

            \item $-1.75$

            \item $7.06$

            \item $-5.9$

        \end{enumerate*}

    \item Para cada inciso del ejercicio anterior, realice la conversión
        inversa (es decir, de Punto Fijo a expresión decimal) e indique el
        \textbf{error de precisión cometido} (la diferencia entre el número
        original y el representado).

    \item Los siguientes números están representados en \emph{Punto Fijo
        y complemento a dos en 8 bits, con cuatro bits para la parte entera y
        4 para la parte fraccionaria}. Indique en cada caso su expresión decimal:

        \begin{enumerate*}[itemjoin=\hspace{2em}]

            \item 0x41

            \item 0xF8

            \item 0xA3

        \end{enumerate*}

    \item Dados los siguientes números representados en \emph{Punto Fijo y
        Complemento a 2, con 4 bits para la parte entra y 4 bits para la parte
        fraccionaria}, efectuar las siguientes sumas y determinar cuales de ellas
        producen \textbf{overflow}:

        \begin{enumerate*}[itemjoin=\hspace{2em}]

            \item $1000.1010 + 1100.0110$

            \item $0001.0000 + 1000.0001$

            \item $0111.1100 + 0111.0010$

        \end{enumerate*}

\end{enumerate}

\end{document}
